\documentclass{beamer}

% select theme
\usetheme{Boadilla}
\usecolortheme{beaver}

\usepackage{kotex}
\usepackage{fancyvrb}
\usepackage{color}

% you need to generate pyg.tex by
% pygmentize -O full -f latex hello.c
\input{pyg.tex}



% syntax highlighted source code 넣는법:
% http://ubuntuforums.org/showthread.php?t=790610
% pygmentize 명령어가 실행될 수 있도록 하자.
% bash에 그냥 pygmentize라고 치면 어느 패키지를 설치해야 하는지 알려줌.
% then...
% 색깔 명령어가 define될 수 있도록... hello.c를 만든 다음,
% pygmentize -O full -f latex hello.c
% 를 돌리면 무엇을 include해야 하는지, define이 뭐가 필요한지 쭉
% stdout으로 출력해줌.
% pygmentize -f latex hello.c 로 나오는 verbatim문을 긁어붙이면 완성.
\begin{document}

% title slide
\begin{frame}
	\title{슬라이드 제목}
	\author{Force Core}
	\date{2010-05-06(목)}
	\titlepage
\end{frame}



% outline slide
\section*{Outline}
\begin{frame}
\tableofcontents
\end{frame}



\section{간단한 슬라이드}

\subsection{hello world!}

% 그냥 글자만 있는 슬라이드
\begin{frame}
	Hello World!
\end{frame}

% 제목도 들어갔다.
\begin{frame}
	\frametitle{제목도 넣기}
	이제 제목이 들어갔다.
	내용도 있음
\end{frame}

% 부제목이 있는 슬라이드
\begin{frame}
	\frametitle{제목도 넣기}
	\framesubtitle{부제목 넣었다}
	부제목을 넣은 슬라이드
\end{frame}

% with bullets
\begin{frame}
	\frametitle{Bullet이 있는 Slide}
	\begin{itemize}
		\item item1
		\item item2
		\item item3
	\end{itemize}
\end{frame}

\begin{frame}
	\frametitle{정의를 하는 슬라이드}
	\begin{definition}[영어 단어 Stream]
		\begin{itemize}
			\item Bach - 흐름, 내, 시내, 개울
			\item 유출, 분류
			\item 흐르다, 흘러가다, 흘러 나오다
			\item 잇달아 나오다, 끊임없이 게속되다
		\end{itemize}
	\end{definition}
\end{frame}

\begin{frame}
	% \begin{figure}
	% \includegraphics[width=0.3\columnwidth]{jpg/bach_shades.jpg}
	% \end{figure}
	% 그림 넣은 슬라이드 \\
	``It's easy to play any musical instrument: all you have to do is touch the right key at the right time and the instrument will play itself.''
	\footnote{\url{http://thinkexist.com/quotation/it-s_easy_to_play_any_musical_instrument-all_you/13822.html}}
\end{frame}

\begin{frame}
	\frametitle{칼럼 넣은 슬라이드}
	\begin{columns}
	\begin{column}{0.5\textwidth}
		\begin{figure}
		\includegraphics[width=0.9\columnwidth]{svg/Stdstreams-notitle}
		\end{figure}
	\end{column}
	\begin{column}{0.5\textwidth}
		\begin{itemize}
			\item stdout
			\item stderr
			\item stdin
		\end{itemize}
	\end{column}
	\end{columns}
\end{frame}


\subsection{서브섹션을 가르면 진도표에 반영됨}

\begin{frame}
	\frametitle{Caeser Cypher II}
	{\em \Large Practice1 - Caeser Cypher II}

	\vspace{5mm}
	중간 타이틀로서 나름 깔끔한듯.
	part기능에는 좀 어울리지 않아서 손으로 직접... 읔;;
\end{frame}

\section{다른 섹션}

\begin{frame}[containsverbatim]
	\frametitle{그냥 verbatim}
	터미널의 출력 따위
	\begin{verbatim}
	$ chmod -x a.out
	$ chmod +x a.out
	$ chmod -r test.c
	$ chmod +r test.c
	$ chmod -w test.c
	$ chmod +w test.c
	\end{verbatim}
\end{frame}

\subsection{Practice2 - chmod}

\begin{frame}
	\frametitle{Practice2 - chmod}
	{\em \Large Practice2 - chmod}

	\vspace{5mm}
	set, unset, get 함수를 완성하여 chmod를 따라해보자.
	set, unset, get함수에는 if문이 필요 없다.
\end{frame}



\end{document}

