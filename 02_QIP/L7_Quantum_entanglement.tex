\documentclass[9pt]{beamer}
\usepackage{kotex}
\usepackage{amsfonts,amssymb,amsthm}
\usepackage[dvipsnames]{xcolor}
\usepackage{xcolor}
\usepackage{etoolbox}
\usepackage{braket}
\usepackage{qcircuit}

%## color
\definecolor{customBlack}{HTML}{3B4252}
\definecolor{customBlackGrey}{HTML}{434C5e}
\definecolor{cuatomGrey}{HTML}{4C566A} 
\definecolor{customWhite}{HTML}{ECEFF4} 
\definecolor{customBlue}{HTML}{6082B6}  
\definecolor{customRed}{HTML}{BF616A}
\definecolor{vividauburn}{rgb}{0.58, 0.15, 0.14}


%## Theme & custom
% \usetheme{metropolis}           % Use metropolis theme
% \metroset{block=fill}
\usetheme{moloch} % modern fork of the metropolis theme
\molochset{block=fill}
\setbeamersize{text margin left=5mm, text margin right=5mm}
\setbeamercolor{palette primary}{bg=customBlack}
\setbeamercolor{alerted text}{fg=customRed}
\setbeamercolor{itemize item}{fg=customBlue}
\setbeamercolor{enumerate item}{fg=customBlue}


%## font
\usefonttheme[onlymath]{serif}
% \setbeamerfont{normal text}{size=\small}
% \setbeamerfont{math text}{size=\tiny}


%## Theorem title, numbering
\makeatletter
\setbeamertemplate{theorem begin}
{%
\begin{\inserttheoremblockenv}
{%
\inserttheoremheadfont
\inserttheoremname
\ifx\inserttheoremaddition\@empty\else\ of\ \inserttheoremaddition\fi%
\inserttheorempunctuation
}%
}
\setbeamertemplate{theorem end}{\end{\inserttheoremblockenv}}
\makeatother
\setbeamertemplate{theorems}[numbered]  


%## Custom block
\setbeamercolor{block title}{bg=customBlue, fg=white}
\setbeamercolor{block body}{bg=customWhite, fg=customBlack}
\setbeamercolor{block title alerted}{%
    use={block title, alerted text},
    bg=customRed,
    fg=white
}
\setbeamercolor{block body alerted}{%
    use={block title, alerted text},
    bg=customWhite,
    fg=customBlack
}
\AtBeginEnvironment{definition}{%
    \setbeamercolor{block title}{fg=white,bg=customBlackGrey}
    \setbeamercolor{block body}{fg=customBlack, bg=customWhite}
}
\AtBeginEnvironment{theorem}{%
    \setbeamercolor{block title}{fg=white,bg=customBlackGrey}
    \setbeamercolor{block body}{fg=customBlack, bg=customWhite}
}
\AtBeginEnvironment{corollary}{%
    \setbeamercolor{block title}{fg=white,bg=customBlackGrey}
    \setbeamercolor{block body}{fg=customBlack, bg=customWhite}
}
\AtBeginEnvironment{lemma}{%
    \setbeamercolor{block title}{fg=white,bg=customBlackGrey}
    \setbeamercolor{block body}{fg=customBlack, bg=customWhite}
}


%! Useful command
\renewcommand{\Pr}{\text{Pr}}
% $\ast$ \underline{Proof}:
%\checkmark \underline{meaning}:

\title{7. Quantum entanglement}
\date{\today}
\author{Vaughan Sohn}
% \institute{Centre for Modern Beamer Themes}


\begin{document}
    %#################################### 
    \maketitle
    
    %#################################### 
    \begin{frame}
        \frametitle{Contents}
        \tableofcontents
    \end{frame}

    %#################################### 
    %pure state가 주어졌을 때 entangled state인지 아닌지 파악하는 방법
    \begin{section}{Entanglement in pure state}
        \begin{frame}
            \frametitle{Product state}
            Pure state에서 entanglement를 판단하는 대표적인 기준은 composite system의 state를 \textbf{product state}의 형태로 표현이 가능한지 확인하는 것이다.
            \begin{itemize}
                \item product state: $\ket{\psi}_{AB} = \ket{\psi}_A \otimes \ket{\psi}_B$
                \item entangled state: $\ket{\psi}_{AB} \ne \ket{\psi}_A \otimes \ket{\psi}_B$
            \end{itemize}
            \vspace{0.4cm}
            Example:
            \begin{itemize}
                \item 다음 state는 entangled state인가?
                \begin{equation*}
                    \frac{1}{\sqrt 2} \left ( \ket{00} + \ket{01} + \ket{10} + \ket{11} \right)
                \end{equation*}
                \item 다음 state는 entangled state인가?
                \begin{equation*}
                    \frac{1}{\sqrt 2} \left ( \ket{00} + \ket{01} + \ket{10} - \ket{11} \right)
                \end{equation*} 
            \end{itemize}
            \begin{block}{Question}
                이 방법보다 더 간단한 방법은 없을까? \\
                $\Rightarrow$ \textit{Schmidt decomposition!}
            \end{block}
        \end{frame}
        
        \begin{frame}
            \frametitle{Schmidt decomposition}
            \begin{theorem}[Schmidt decomposition]
                Schmidt decomposition은 다음과 같이 정의된다.
                \begin{equation*}
                    \ket{\psi} = \sum_{ij} c_{ij} \ket{i} \ket{j} = \sum_{k=1}^d D_{k} \ket{u_k} \ket{v_k}
                \end{equation*}
                where $\ket{\psi}_{AB} \in \mathcal H_A \otimes \mathcal H_d$, $dim(H_A \otimes H_b) = d^2$.
            \end{theorem}
            \vspace{0.4cm}
            Schmidt coefficient의 \textbf{rank}를 확인하여 entangled인지 구분할 수 있다.
            \begin{itemize}
                \item product state: $d=1$
                \item entangled state: $d>1$
            \end{itemize}
        \end{frame}

        \begin{frame}
            \frametitle{Schmidt decomposition}
            $\ast$ \underline{Proof}:\\
            \vspace{0.2cm}
            Schmidt decomposition은 다음과 같이 quantum state의 coefficient를 \alert{대각화가 가능한 $d\times d$ matrix의 원소}로 생각하는 것에서부터 출발한다.
            \begin{equation*}
                c_{ij} = [C]_{ij} = [UDV]_{ij}
            \end{equation*}
            따라서 이 표현을 대신 대입하게되면, 대각행렬 $D$의 원소들만을 사용하여 quantum state를 새로운 basis에 대해 표현할 수 있다.
            \begin{align*}
                |\psi\rangle & =\sum_{i j} [U D V]_{i j}\ket i \ket j \\
                &=\sum_{i j k} u_{i k} D_{k k} v_{k j}|i\rangle|j\rangle \\
                &=\sum_k D_k \underbrace{\sum_i u_{i k}|i\rangle}_{\left|u_k\right\rangle} \underbrace{\sum_j v_{k j}|j\rangle}_{\left|v_k\right\rangle} \\
                &=\sum_{k=1}^d D_k\left|u_k\right\rangle\left|v_k\right\rangle.
            \end{align*}
        \end{frame}

        \begin{frame}
            \frametitle{LU equivalent}
            \begin{definition}[LU equivalent]
                두 $n$-qubit state $\ket \psi$와 $\ket \phi$가 \textbf{LU equivalent}라면 다음을 만족하는 어떤 \alert{local unitary} $U_1, U_2, \cdots, U_n$가 존재한다. 
                \begin{equation*}
                    \ket{\phi} = (U_1 \otimes U_2 \otimes \cdots \otimes U_n) \ket\psi
                \end{equation*}
            \end{definition}
            \vspace{0.2cm}
            $\Rightarrow$ 만약 어떤 임의의 state $\ket{\psi}$가 product state와 LU equivalent라면 $\ket{\psi}$ \textbf{역시} product state이며, entangled state와 LU equivalent라면 $\ket{\psi}$ \textbf{역시} entangled state이다.\\\
            \vspace{0.2cm}
            
            Example: 
            \begin{itemize}
                \item LU equivalent인지 확인하고 이로부터 entangled state인지 판단하라.
                \begin{equation*}
                    \frac{1}{\sqrt 2} \left ( \ket{00} + \ket{01} + \ket{10} - \ket{11} \right), \qquad \frac{\ket{00} + \ket{11}}{ \sqrt 2}
                \end{equation*}
                \item LU equivalent인지 확인하고 이로부터 product state인지 판단하라.
                \begin{equation*}
                    \frac{1}{\sqrt 2} \left ( \ket{00} + \ket{01} + \ket{10} + \ket{11} \right), \qquad \ket{00}
                \end{equation*}
            \end{itemize}
        \end{frame}

        \begin{frame}
            \frametitle{LU equivalent and Schmidt coefficient\footnote{Schmidt rank 또한 entanglement의 양을 분석하기 위해 사용되므로 LU equivalent와 관계가 있다.}}
            \begin{block}{Note}
                만약 두 상태가 \textbf{동일한 Schmidt coefficient}를 가진다면, LU equivalent이다.
            \end{block}
            \vspace{0.2cm}
            다음과 같이 Schmidt decomposition으로 표현된 두 상태를 가정하자.
            \begin{itemize}
                \item $\ket{\psi} = D_1 \ket{00} + D_2 \ket{11}$
                \item $\ket{\phi} = D_1 \ket{uv} + D_2 \ket{u^\perp v^\perp}$
            \end{itemize}
            \vspace{0.2cm}
            그렇다면, 다음의 unitary가 존재함을 쉽게 추측할 수 있다.
            \begin{itemize}
                \item $U \ket{u} = \ket 0, U\ket{u^\perp} = 1$
                \item $V \ket{v} = \ket 0, V\ket{v^\perp} = 1$
            \end{itemize}
            \vspace{0.2cm}
            따라서 두 상태는 LU equivalent이다.
            \begin{equation*}
                \ket{\psi} = (U \otimes V) \ket{\phi}
            \end{equation*}
        \end{frame}


        \begin{frame}
            \frametitle{Reduced state and LU equivalent}
            (recap) Composite system은 \textbf{partial trace}를 이용하여 각 composed system에 대한 reduced state를 나타낼 수 있다.
            \begin{itemize}
                \item $\rho^A = \text{tr}_B [\rho^{AB}] = \sum_i \bra i_B \rho^{AB} \ket i_B$
                \item $\rho^B = \text{tr}_A [\rho^{AB}] = \sum_i \bra i_A \rho^{AB} \ket i_A$
            \end{itemize}
            \vspace{0.4cm}
            LU equivalent한 다음 두 상태를 가정하자.
            \begin{itemize}
                \item $\ket{\psi} = D_1 \ket{00} + D_2 \ket{11}$
                \item $\ket{\phi} = D_1 \ket{uv} + D_2 \ket{u^\perp v^\perp}$
            \end{itemize}
            \begin{equation*}
                \ket{\psi} = (U \otimes V) \ket{\phi}
            \end{equation*}

            \vspace{0.2cm}
            각 상태에 대해 reduced state를 구하면, 다음과 같다.
            \begin{equation*}
                \ket{\psi} = \begin{cases}
                    \rho^A_\psi = D_1^2 \ket 0 \bra 0 + D_2^2 \ket 1 \bra 1\\
                    \rho^B_\psi = D_1^2 \ket 0 \bra 0 + D_2^2 \ket 1 \bra 1
                \end{cases},\quad \ket{\phi} = \begin{cases}
                    \rho^A_\phi = D_1^2 \ket u \bra u + D_2^2 \ket {u^\perp} \bra {u^\perp}\\
                    \rho^B_\phi = D_1^2 \ket v \bra v + D_2^2 \ket {v^\perp} \bra {v^\perp}
                \end{cases}
            \end{equation*}
            $\Rightarrow$ Spectral decomposition으로 표현된 $\rho^A, \rho^B$가 동일한 eigenvalue($\lambda_i \triangleq D_i^2$)를 가지는 것을 알 수 있다.\footnote{$\rho^A_\psi, \rho^B_\psi$는 완전히 equivalent한 state, $\rho^A_\phi, \rho^B_\phi$는 Schmidt coefficient가 동일하므로 LU equivalent.}
        \end{frame}

        \begin{frame}
            \frametitle{Example}
            다음 state가 entangled state인지 아닌지 다양한 방법\footnote{product, Schmidt decomposition, LU-equivalent}으로 해결해보자.
            \begin{itemize}
                \item Example 1:
                \begin{equation*}
                    \frac{1}{\sqrt 2} \left ( \ket{00} + \ket{01} + \ket{10} + \ket{11} \right)
                \end{equation*}
                \item Example 2:
                \begin{equation*}
                    \frac{1}{\sqrt 3} \left ( \ket{00} + \ket{01} + \ket{10} \right)
                \end{equation*}
            \end{itemize}
        \end{frame}

        \begin{frame}
            \frametitle{Example}
            (Example 2) LU equivalent와 reduced matrix를 이용하여 해결하는 solution:\\
            주어진 state가 다음과 같이 Schmidt decomposition으로 표현할 수 있다고 하자.
            \begin{equation*}
                \ket{\psi} = \frac{1}{\sqrt 3}(\ket{00} + \ket{01} + \ket{10}) = \sum_k D_k \ket{u_k} \ket{v_k}
            \end{equation*}
            그러면 각각의 system $A, B$에 local operator를 가하여 동일한 basis $\{\ket k\}$로 나타낼 수 있으며, 이 state와는 LU-equivalent 관계이다.
            \begin{equation*}
                \ket{\psi'} = \sum_k D_k \ket{k} \ket{k} = (U \otimes V) \ket{\psi}
            \end{equation*}
            $\ket{\psi}$에 대한 reduced state는 각각 다음과 같다.
            \begin{equation*}
                \rho^A = \sum D_k^2 \ket{u_k} \bra{u_k} = U\left(\sum_k D_k^2 \ket{k} \bra{k}\right)U^\dagger
            \end{equation*}
            \begin{equation*}
                \Rightarrow \rho^A = \frac{2}{3} \ket + \bra + + \frac{1}{3} \ket 0 \bra 0= U\begin{pmatrix} \lambda_1 & 0 \\ 0 & \lambda_2 \end{pmatrix}U^\dagger
            \end{equation*}
            이때, $\lambda_1 > 0$, $\lambda_2 > 0$이므로 $\ket{\psi}$는 entangled state이다.$\Box$
        \end{frame}

        \begin{frame}
            \frametitle{Canonical form of Two-qubit state}
            Schmidt decomposition을 이용하면 canonical form을 정의할 수 있다.
            \begin{definition}[Canonical form of two-qubit state]
                다음과 같이 표현되는 two-qubit state는 어떤 two-qubit state $\ket{\varphi}$와도 LU-equivalent하도록 만드는 coefficient $\alpha, \beta$를 가진다.
                \begin{align*}
                    \ket \psi &= \alpha \ket{00} + \beta \ket{11}\\
                                &= \cos \theta \ket{00} + \sin \theta \ket{11},\qquad (0 \le \theta \le \pi/4)
                \end{align*}
            \end{definition}
            \vspace{0.4cm}
            $\ast$ (why?) : $\{\ket0, \ket 1\}$은 single qubit state의 대표적인 basis이며, 위의 canonical form은 대표적인 basis를 이용한 Schmidt decomposition의 형태이다.\\
            \vspace{0.2cm}
            \textit{Local unitary}를 이용하면 어떤 형태의 Schmidt decomposition이든 위와 같은 형태로 변환시킬 수 있기 때문에, 어떠한 상태와도 LU-equivalent 할 수 있다.
            \begin{itemize}
                \item $\ket{u_1} = U\ket{0},\quad \ket{u_2} = U\ket{1}$
                \item $\ket{v_1} = V\ket{0},\quad \ket{v_2} = V\ket{1}$
            \end{itemize}
            \vspace{0.2cm}
            \begin{equation*}
                \ket \varphi =  \alpha \ket{u_1v_1} + \beta \ket{u_2 v_2} = (U \otimes V)(\alpha \ket{00} + \beta \ket{11}) \approx^{LU} \ket{\psi}
            \end{equation*}
        
        \end{frame}
    \end{section}
    
    %#################################### 
    %mixed state가 주어졌을 때 entangled state인지 아닌지 파악하는 방법
    \begin{section}{Entanglement in mixed state}
        \begin{frame}
            \frametitle{Separable state}
            Mixed state에서 entanglement를 판단하는 대표적인 기준은 composite system의 density matrix가 \textbf{separable state}의 형태로 표현이 가능한지 확인하는 것이다.
            \begin{definition}[Separable state]
                A separable state can be prepared by LOCC(Local Operation and Classical Communication)
                \begin{equation*}
                    \rho^{AB} = \sum_i p(i) \left( \rho_i^A \otimes \rho_i^B \right)
                \end{equation*}
                \begin{itemize}
                    \item LO: 각 system $A$, $B$에서 $\rho_i$를 준비하기 위해 사용하는 연산
                    \item CC: mixed state의 확률분포 $\sim p$를 공유하기 위한 communication\footnote{TODO: $A$, $B$가 각각 $p(i)$의 확률로 $\rho_i^A$를 준비하는게 맞는지 확인 필요}
                \end{itemize}
            \end{definition}
        \end{frame}
        
        \begin{frame}
            \frametitle{Non-separable state: entangled state}
            \begin{definition}[Non-separable state]
                A separable state cannot be prepared by LOCC
                \begin{equation*}
                    \rho^{AB} \ne \sum_i p(i) \left( \rho_i^A \otimes \rho_i^B \right)
                \end{equation*}
                \vspace{-0.2cm}
            \end{definition}
            \begin{itemize}
                \item Example 1 : $\epsilon$이 1에 가까울수록 separable state에 가까워 짐.\footnote{$I/4$가 separable이므로}
                \begin{equation*}
                    \rho = (1-\epsilon) \ket{\psi_{ent}} \bra{\psi_{ent}} + \epsilon \  \frac{I}{4}
                \end{equation*}
                \item Example 2 :
                \begin{equation*}
                    \rho = \frac{1}{2} \ket{00} \bra{00} +  \frac{1}{2} \ket{11} \bra{11}
                \end{equation*}
                \begin{equation*}
                    \text{(sol.)}\quad  \rho=\frac{1}{2}(|0\rangle\langle 0| \otimes|0\rangle\langle 0|)+\frac{1}{2}(|1\rangle\langle 1| \otimes|1\rangle\langle 1|)
                \end{equation*}
                \item Example 3\footnote{부호가 다른 두 entanglement state들의 mixed state라서 product로 표현할 수 없는 state들이 소거된다.} : 
                \begin{equation*}
                    \rho = \frac{1}{2} \ket{\Phi^+} \bra{\Phi^+} +  \frac{1}{2} \ket{\Phi^-} \bra{\Phi^-}
                \end{equation*}
                \begin{equation*}
                    \text{(sol.)}\quad  \rho=\frac{1}{2}(|0\rangle\langle 0| \otimes|0\rangle\langle 0| + |1\rangle\langle 1| \otimes|1\rangle\langle 1|)
                \end{equation*}
            \end{itemize}
            
        \end{frame}

        \begin{frame}
            \frametitle{Example}
            \begin{itemize}
                \item Example 4 : 
                \begin{equation*}
                    \rho = \frac{1}{4} \left( \ket{\Phi^+} \bra{\Phi^+}  +  \ket{\Phi^-} \bra{\Phi^-}  + \ket{\Psi^+} \bra{\Psi^+}  +  \ket{\Psi^-} \bra{\Psi^-} \right)
                \end{equation*}
                \item Example 5 :
                \begin{equation*}
                    \rho = \frac{1}{3} \left( \ket{\Phi^+}\bra{\Phi^+} + \ket{\Phi^-} \bra{\Phi^-} + \ket{\Psi^+} \bra{\Psi^+} \right)
                \end{equation*}
            \end{itemize}
        \end{frame}

        \begin{frame}
            \frametitle{PPT Criteria}
            Mixed state의 entanglement를 확인하는 또 하나의 방법을 바로 \textbf{PPT Criteria}라고 한다. 
            \begin{theorem}[Positive Partial Transpose contidion]
                만약 $\rho$가 separable state라면, $\rho$를 이루는 system중에서 일부 system만 \alert{transpose}를 취하더라도 여전히 quantum state, 즉 \textbf{positive matrix}여야한다.
                \begin{equation*}
                    \rho^{T_B} = \sum_i p(i) \ket{e_i}\bra{e_i} \otimes \left(\ket{f_i} \bra{f_i}\right)^T \ge 0,
                \end{equation*}
                where $\rho = \sum_i p(i) \ket{e_i} \bra{e_i} \otimes \ket{f_i} \bra{f_i}$.
            \end{theorem}
            \vspace{0.2cm}
            \begin{itemize}
                \item Partial transpose: 특정 system에 대해서만 transpose를 취하는 연산
                \item Partial transpose in matrix: 
                \begin{itemize}
                    \item $T_A$: Block 1과 Block 4 내부의 요소가 각각 전치된다.
                    \item $T_B$: Block 2와 Block 3 내부의 요소가 각각 전치된다.
                \end{itemize}
                \begin{equation*}
                    \rho =
                    \begin{bmatrix}
                    \text{Block 1 (00)} & \text{Block 2 (01)} \\
                    \text{Block 3 (10)} & \text{Block 4 (11)}
                    \end{bmatrix}
                \end{equation*}
                \item Positive matrix: non-negative eigenvalue만을 가지는 행렬
            \end{itemize}
            
        \end{frame}

        \begin{frame}
            \frametitle{Example}
            Example 1 (Bell state) : \\
            $\rho = \ket{\Phi^+} \bra{\Phi^+}$의 matrix representation은 다음과 같다.
            \begin{equation*}
                \rho = \ket{\Phi^+} \bra{\Phi^+} = \frac{1}{2} \begin{pmatrix}
                    1 & 0 & 0 & 1 \\
                    0 & 0 & 0 & 0 \\
                    0 & 0 & 0 & 0 \\
                    1 & 0 & 0 & 1
                \end{pmatrix}, \qquad \text{eig}(\rho) : 1/2, 0
            \end{equation*}
            반면, 이 state의 partial transpose는 다음과 같이 \textbf{negative eigenvalue}를 가지기 때문에 entangled state이다.
            \begin{equation*}
                \rho^{T_B} = \frac{1}{2} \begin{pmatrix}
                    1 & 0 & 0 & 0 \\
                    0 & 0 & 1 & 0 \\
                    0 & 1 & 0 & 0 \\
                    0 & 0 & 0 & 1
                \end{pmatrix}, \qquad \text{eig}(\rho) : -1/2, 1/2
            \end{equation*}
        \end{frame}

        \begin{frame}
            \frametitle{Example}
            Example 2 : \\
            다음과같이 Bell state와 pure state의 mixed state로 표현된 $\rho$를 $p$가 얼마일때를 기준으로 entangled / separable state로 나뉘는가?
            \begin{equation*}
                \rho_p = (1-p) \ket{\Phi^+}\bra{\Phi^+} + p \frac{I}{4}
            \end{equation*}
            $\Rightarrow$ (hint) 기준이 되는 지점은 partial trace의 \textit{minimum eigenvalue}가 음수가 되는 지점!
            \begin{equation*}
                \rho^{T_B} = (1-p) \underbrace{\left(\ket{\Phi^+}\bra{\Phi^+}\right)^{T_B}}_{\lambda_{\min} = -1/2} + p \underbrace{\left(\frac{I}{4}\right)^{T_B}}_{\lambda_{\min} = 1/4}
            \end{equation*}
            따라서 minimum eigenvalue로부터 얻은 기준은 $\boxed{p=2/3}$ 이다.
            \begin{align*}
                \lambda^\ast &= \min_{i} \braket{v_i|\rho^{T_B} | v_i} \\
                            &= (1-p) \left(-\frac{1}{2}\right) + \frac{p}{4} < 0
            \end{align*}
        \end{frame}
    \end{section}

    \begin{frame}
        \frametitle{Summary: condition of entanglement}
        \begin{block}{Summary}
            \begin{itemize}
                \item pure state
                \begin{itemize}
                    \item (definition) product state인지 아닌지 확인한다.
                    \begin{equation*}
                        \text{(sep)} \qquad \ket{\psi}^{AB} = \ket{\psi}^A \otimes \ket{\psi}^B
                    \end{equation*}
                    \item Schmidt decomposition : rank = 1이라면 product, rank $>$ 1이라면 entangled.
                    \item Product state와 LU-equivalent라면 product, entangled state와 LU-equivalent라면 entangled.
                    \item Reduced state의 Schmidt rank = 1이라면 product, rank $>$ 1이라면 entangled.\footnote{확인 필요}
                \end{itemize}
                \item mixed state
                \begin{itemize}
                    \item (definition) separable state인지 아닌지 확인한다.
                        \begin{equation*}
                            \text{(sep)} \qquad \rho^{AB} = \sum_i p(i) \rho_i^A \otimes \rho_i^B
                        \end{equation*}
                    \item PPT Criteria : partial trace가 positive라면 separable, positive가 아니라면 entangled.
                \end{itemize}
            \end{itemize}
        \end{block}
    \end{frame}

    %#################################### 
    %entanglement의 양을 나타내는 방법
    \begin{section}{Quantity of the entanglement}
        \begin{frame}
            \frametitle{Entanglement measure}
            
        \end{frame}

        \begin{frame}
            \frametitle{Distillation of entanglement in pure state}
            
        \end{frame}

        \begin{frame}
            \frametitle{Distillation of entanglement in mixed state}
            
        \end{frame}
    \end{section}


    \begin{frame}
        \frametitle{Summary: quantuntity of entanglement}
        
    \end{frame}

    %#################################### 
    %state가 어떤 state인지에 대한 정보가 없을 때 (arbitrary 안 알려진 상태) entanglement의 양을 나타내는 방법
    \begin{section}{Entanglement witness}
        \begin{frame}
            \frametitle{Positive trace test}
            
        \end{frame}
        
        \begin{frame}
            \frametitle{Feasibleness}
            
        \end{frame}
        
        \begin{frame}
            \frametitle{Example}
            
        \end{frame}
    \end{section}

    \begin{frame}
        \frametitle{Summary: entanglement witness of arbitrary quantum state}
        
    \end{frame}

    %#################################### 
    \begin{section}{General theory of  Entanglement witness}
        \begin{frame}
            \frametitle{Convexity on separable state set}
            
        \end{frame}
        
        \begin{frame}
            \frametitle{Quantum channel}
            
        \end{frame}

        \begin{frame}
            \frametitle{Another definition of entanglement using Channel}
            
        \end{frame}

        \begin{frame}
            \frametitle{Choi-Jamiolkowski isomorphism}
            
        \end{frame}
    \end{section}

    %#################################### 
    \begin{frame}{Summary}
        \begin{block}{Summary}
            \begin{itemize}
                \item 
            \end{itemize}
        \end{block}
    \end{frame}


    %#################################### 
    \begin{frame}{References}
        \begin{itemize}
            \item Lecture notes for EE547: Introduction to Quantum Information Processing (Fall 2024)
        \end{itemize}
        \vspace{6cm}
    \end{frame}


\end{document}