\documentclass[9pt]{beamer}
\usepackage{kotex}
\usepackage{amsfonts,amssymb,amsthm}
\usepackage[dvipsnames]{xcolor}
\usepackage{xcolor}
\usepackage{etoolbox}
\usepackage{braket}
\usepackage{qcircuit}

%## color
\definecolor{customBlack}{HTML}{3B4252}
\definecolor{customBlackGrey}{HTML}{434C5e}
\definecolor{cuatomGrey}{HTML}{4C566A} 
\definecolor{customWhite}{HTML}{ECEFF4} 
\definecolor{customBlue}{HTML}{6082B6}  
\definecolor{customRed}{HTML}{BF616A}
\definecolor{vividauburn}{rgb}{0.58, 0.15, 0.14}


%## Theme & custom
% \usetheme{metropolis}           % Use metropolis theme
% \metroset{block=fill}
\usetheme{moloch} % modern fork of the metropolis theme
\molochset{block=fill}
\setbeamersize{text margin left=5mm, text margin right=5mm}
\setbeamercolor{palette primary}{bg=customBlack}
\setbeamercolor{alerted text}{fg=customRed}
\setbeamercolor{itemize item}{fg=customBlue}
\setbeamercolor{enumerate item}{fg=customBlue}


%## font
\usefonttheme[onlymath]{serif}
% \setbeamerfont{normal text}{size=\small}
% \setbeamerfont{math text}{size=\tiny}


%## Theorem title, numbering
\makeatletter
\setbeamertemplate{theorem begin}
{%
\begin{\inserttheoremblockenv}
{%
\inserttheoremheadfont
\inserttheoremname
\ifx\inserttheoremaddition\@empty\else\ of\ \inserttheoremaddition\fi%
\inserttheorempunctuation
}%
}
\setbeamertemplate{theorem end}{\end{\inserttheoremblockenv}}
\makeatother
\setbeamertemplate{theorems}[numbered]  


%## Custom block
\setbeamercolor{block title}{bg=customBlue, fg=white}
\setbeamercolor{block body}{bg=customWhite, fg=customBlack}
\setbeamercolor{block title alerted}{%
    use={block title, alerted text},
    bg=customRed,
    fg=white
}
\setbeamercolor{block body alerted}{%
    use={block title, alerted text},
    bg=customWhite,
    fg=customBlack
}
\AtBeginEnvironment{definition}{%
    \setbeamercolor{block title}{fg=white,bg=customBlackGrey}
    \setbeamercolor{block body}{fg=customBlack, bg=customWhite}
}
\AtBeginEnvironment{theorem}{%
    \setbeamercolor{block title}{fg=white,bg=customBlackGrey}
    \setbeamercolor{block body}{fg=customBlack, bg=customWhite}
}
\AtBeginEnvironment{corollary}{%
    \setbeamercolor{block title}{fg=white,bg=customBlackGrey}
    \setbeamercolor{block body}{fg=customBlack, bg=customWhite}
}
\AtBeginEnvironment{lemma}{%
    \setbeamercolor{block title}{fg=white,bg=customBlackGrey}
    \setbeamercolor{block body}{fg=customBlack, bg=customWhite}
}


%! Useful command
\renewcommand{\Pr}{\text{Pr}}
% $\ast$ \underline{Proof}:
%\checkmark \underline{meaning}:

\title{2. Basic Quantum Computing with Deutsch Algorithm}
\date{\today}
\author{Vaughan Sohn}
% \institute{Centre for Modern Beamer Themes}


\begin{document}
    %#################################### 
    \maketitle
    

    %#################################### 
    \begin{frame}
        \frametitle{Contents}
        \tableofcontents
    \end{frame}

    %#################################### 
    \begin{frame}
        \frametitle{Overview of Quantum computation}
        
    \end{frame}

    %#################################### 
    \begin{section}{Deutsch Algorithm}
        \begin{frame}
            \frametitle{Deutsch Problem}
            
        \end{frame}

        \begin{frame}
            \frametitle{Solutions of Deutsch Problem}
            
        \end{frame}
    \end{section}

    %#################################### 
    \begin{section}{Key Concept: Discrimination}
        \begin{frame}
            \frametitle{Discrimination between quantum states}
            
        \end{frame}

        \begin{frame}
            \frametitle{Discrimination between unitaries}
            
        \end{frame}

        \begin{frame}
            \frametitle{Discrimination in Deutsch Algorithm}
            \framesubtitle{Deutsch Algorithm is only Discrimination of unitaries!}
            
        \end{frame}
    \end{section}

    %#################################### 
    \begin{section}{Key Concept: Phase estimation}
        \begin{frame}
            \frametitle{t-qubit Deutsch Algorithm}

        \end{frame}

        \begin{frame}
            \frametitle{Quantum Fourier Transform}

        \end{frame}

        \begin{frame}
            \frametitle{Phase estimation in Deutsch Algotirhm}
            \framesubtitle{Deutsch Algorithm requires Phase Estimation to get the superposition result!}
            
        \end{frame}
    \end{section}

    %#################################### 
    \begin{section}{Universial gate set}
        \begin{frame}
            \frametitle{Universial gate set}
            
        \end{frame}

        \begin{frame}
            \frametitle{Prerequisite: bloch sphere and rotation gate}
            
        \end{frame}

        \begin{frame}
            \frametitle{Controlled unitary decomposition}
            
        \end{frame}

        \begin{frame}
            \frametitle{Example: Controlled Y gate}
            
        \end{frame}

        \begin{frame}
            \frametitle{Approximating quantum circuits via discrete gate set}
            \framesubtitle{Solovay-Kitaev Theorem}
            
        \end{frame}
    \end{section}

    %#################################### 
    \begin{frame}{References}
        
        \begin{itemize}
            \item Lecture notes for EE547: Introduction to Quantum Information Processing (Fall 2024)
        \end{itemize}
        \vspace{6cm}
    \end{frame}

\end{document}