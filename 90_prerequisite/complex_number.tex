\documentclass[9pt]{beamer}
\usepackage{kotex}
\usepackage{amsfonts,amssymb,amsthm}
\usepackage[dvipsnames]{xcolor}
\usepackage{xcolor}
\usepackage{etoolbox}
\usepackage{braket}
\usepackage{qcircuit}

%## color
\definecolor{customBlack}{HTML}{3B4252}
\definecolor{customBlackGrey}{HTML}{434C5e}
\definecolor{cuatomGrey}{HTML}{4C566A} 
\definecolor{customWhite}{HTML}{ECEFF4} 
\definecolor{customBlue}{HTML}{6082B6}  
\definecolor{customRed}{HTML}{BF616A}
\definecolor{vividauburn}{rgb}{0.58, 0.15, 0.14}


%## Theme & custom
% \usetheme{metropolis}           % Use metropolis theme
% \metroset{block=fill}
\usetheme{moloch} % modern fork of the metropolis theme
\molochset{block=fill}
\setbeamersize{text margin left=5mm, text margin right=5mm}
\setbeamercolor{palette primary}{bg=customBlack}
\setbeamercolor{alerted text}{fg=customRed}
\setbeamercolor{itemize item}{fg=customBlue}
\setbeamercolor{enumerate item}{fg=customBlue}


%## font
\usefonttheme[onlymath]{serif}
% \setbeamerfont{normal text}{size=\small}
% \setbeamerfont{math text}{size=\tiny}


%## Theorem title, numbering
\makeatletter
\setbeamertemplate{theorem begin}
{%
\begin{\inserttheoremblockenv}
{%
\inserttheoremheadfont
\inserttheoremname
\ifx\inserttheoremaddition\@empty\else\ of\ \inserttheoremaddition\fi%
\inserttheorempunctuation
}%
}
\setbeamertemplate{theorem end}{\end{\inserttheoremblockenv}}
\makeatother
\setbeamertemplate{theorems}[numbered]  


%## Custom block
\setbeamercolor{block title}{bg=customBlue, fg=white}
\setbeamercolor{block body}{bg=customWhite, fg=customBlack}
\setbeamercolor{block title alerted}{%
    use={block title, alerted text},
    bg=customRed,
    fg=white
}
\setbeamercolor{block body alerted}{%
    use={block title, alerted text},
    bg=customWhite,
    fg=customBlack
}
\AtBeginEnvironment{definition}{%
    \setbeamercolor{block title}{fg=white,bg=customBlackGrey}
    \setbeamercolor{block body}{fg=customBlack, bg=customWhite}
}
\AtBeginEnvironment{theorem}{%
    \setbeamercolor{block title}{fg=white,bg=customBlackGrey}
    \setbeamercolor{block body}{fg=customBlack, bg=customWhite}
}
\AtBeginEnvironment{corollary}{%
    \setbeamercolor{block title}{fg=white,bg=customBlackGrey}
    \setbeamercolor{block body}{fg=customBlack, bg=customWhite}
}
\AtBeginEnvironment{lemma}{%
    \setbeamercolor{block title}{fg=white,bg=customBlackGrey}
    \setbeamercolor{block body}{fg=customBlack, bg=customWhite}
}


%! Useful command
\renewcommand{\Pr}{\text{Pr}}
% $\ast$ \underline{Proof}:
%\checkmark \underline{meaning}:

\title{Complex number}
\date{\today}
\author{Vaughan Sohn}
% \institute{Centre for Modern Beamer Themes}


\begin{document}
    %#################################### 
    \maketitle
    
    %#################################### 
    \begin{frame}
        \frametitle{Contents}
        \tableofcontents
    \end{frame}

    \begin{section}{Euler's equation}
        \begin{frame}
            \frametitle{Euler's equation}
            \textit{Complex number:}
            $$ z = x + iy = re^{i \theta}$$
            where $x = r\cos \theta, y= r\sin \theta$ and $r = \sqrt{x^2 + y^2}, \theta = \tan^{-1}(y/x)$
            \vspace{0.2cm}
            \begin{theorem}[Euler's formular]
                $$
                \begin{aligned}
                r e^{i \theta} & =r \cos \theta+i r \sin \theta \\
                e^{i \theta} & =\cos \theta+i \sin \theta \quad(\text { if } \mathrm{r}=1)
                \end{aligned}
                $$
            \end{theorem}
            \begin{theorem}[Inverse Euler's formular]
                $$
                \begin{aligned}
                & \cos \theta =\frac{1}{2}\left(e^{i \theta }+e^{-i \theta }\right) \\
                & \sin \theta =\frac{1}{2 i}\left(e^{i \theta }-e^{-i \theta }\right)
                \end{aligned}
                $$
            \end{theorem}
        \end{frame}

    \end{section}

    \begin{section}{Some special value}
        \begin{frame}
            \frametitle{Some Special value}
            \begin{itemize}
                \item $e^{i0} = e^{i2\pi} = 1$
                \item $e^{i\pi/2} = e^{-i 3\pi/2} = i$
                \item $e^{i \pi} = e^{i -\pi} = -1$
                \item $e^{i 3\pi/2} = e^{i -\pi/2} = -i$
            \end{itemize}
            
        \end{frame}

    \end{section}


    \begin{section}{Trigonometric function}
        \begin{frame}
            \frametitle{Basic property}
            $$ \sin^2 \theta + \cos ^2  \theta= 1$$
            $$\tan \theta = \frac{\sin \theta}{ \cos \theta}$$
            $$ \sin (-\theta) = - \sin (\theta)$$
            $$ \cos (-\theta) =  \cos (\theta)$$
            $$ \sin \theta = \cos (\theta - \pi/2)$$
            $$ \cos \theta = \sin (\theta + \pi/2)$$
        
            
        
        \end{frame}
        \begin{frame}
            \frametitle{Trigonometric function property}
            \small
            \begin{align*}  \sin 2\alpha &= 2\sin\alpha\cos \alpha \\ \cos 2\alpha &= 2 \cos^2 \alpha -1 \\ &= 1-2\sin^2 \alpha  \\ \tan 2\alpha &= \frac{2\tan \alpha}{1- \tan^2 \alpha} \end{align*}
            
            \begin{align*}  \sin 3\alpha &= 3\sin\alpha - 4\sin^3 \alpha \\ \cos 3\alpha &=  4\cos^3 \alpha -3\cos \alpha  \\ \tan3 \alpha &= \frac{3\tan \alpha - \tan^3 \alpha}{1- 3\tan^2 \alpha} \end{align*}
            
            \begin{align*}  \sin^2 \frac{\alpha}{2} &= \frac{1-\cos \alpha}{2} \\ \cos^2 \frac{\alpha}{2} &= \frac{1+\cos \alpha}{2} \\ \tan^2 \frac{\alpha}{2} &= \frac{1-\cos \alpha}{1+ \cos \alpha} \end{align*}
        \end{frame}
        
        \begin{frame}
            \frametitle{Trigonometric function property}
            \small
            \begin{align*} \sin(\alpha \pm \beta) &= \sin \alpha \cos \beta \pm  \cos \alpha \sin \beta\\\cos(\alpha \pm \beta) &= \cos \alpha \cos \beta \mp \sin \alpha \sin \beta \\ \tan(\alpha \pm \beta) &= \frac{\tan\alpha \pm \tan \beta}{1 \mp \tan \alpha \tan \beta}\end{align*}
                
            \begin{align*}  \sin \alpha \cos \beta &= \frac{1}{2} \{ \sin(\alpha + \beta) + \sin(\alpha - \beta)\} \\ \cos \alpha \sin \beta &= \frac{1}{2} \{ \sin(\alpha + \beta) - \sin(\alpha - \beta)\}\\ \cos \alpha \cos \beta &= \frac{1}{2} \{ \cos(\alpha + \beta) + \cos(\alpha - \beta)\}\\ \sin \alpha \sin \beta &= \frac{1}{2} \{ \cos(\alpha + \beta) - \cos(\alpha - \beta)\}\end{align*}

            \begin{align*}  \sin A + \sin B &= 2\sin\frac{A+B}{2}  \cos \frac{A-B}{2}\\  \sin A - \sin B &= 2\cos\frac{A+B}{2}  \sin \frac{A-B}{2}\\  \cos A + \cos B &= 2\cos\frac{A+B}{2}  \cos \frac{A-B}{2}\\ \cos A - \cos B &= - 2\sin\frac{A+B}{2}  \sin \frac{A-B}{2}\end{align*}
                
            
        
        \end{frame}
    \end{section}

\end{document}