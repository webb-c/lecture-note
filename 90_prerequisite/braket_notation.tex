\documentclass[9pt]{beamer}
\usepackage{kotex}
\usepackage{amsfonts,amssymb,amsthm}
\usepackage[dvipsnames]{xcolor}
\usepackage{xcolor}
\usepackage{etoolbox}
\usepackage{braket}
\usepackage{qcircuit}

%## color
\definecolor{customBlack}{HTML}{3B4252}
\definecolor{customBlackGrey}{HTML}{434C5e}
\definecolor{cuatomGrey}{HTML}{4C566A} 
\definecolor{customWhite}{HTML}{ECEFF4} 
\definecolor{customBlue}{HTML}{6082B6}  
\definecolor{customRed}{HTML}{BF616A}
\definecolor{vividauburn}{rgb}{0.58, 0.15, 0.14}


%## Theme & custom
% \usetheme{metropolis}           % Use metropolis theme
% \metroset{block=fill}
\usetheme{moloch} % modern fork of the metropolis theme
\molochset{block=fill}
\setbeamersize{text margin left=5mm, text margin right=5mm}
\setbeamercolor{palette primary}{bg=customBlack}
\setbeamercolor{alerted text}{fg=customRed}
\setbeamercolor{itemize item}{fg=customBlue}
\setbeamercolor{enumerate item}{fg=customBlue}


%## font
\usefonttheme[onlymath]{serif}
% \setbeamerfont{normal text}{size=\small}
% \setbeamerfont{math text}{size=\tiny}


%## Theorem title, numbering
\makeatletter
\setbeamertemplate{theorem begin}
{%
\begin{\inserttheoremblockenv}
{%
\inserttheoremheadfont
\inserttheoremname
\ifx\inserttheoremaddition\@empty\else\ of\ \inserttheoremaddition\fi%
\inserttheorempunctuation
}%
}
\setbeamertemplate{theorem end}{\end{\inserttheoremblockenv}}
\makeatother
\setbeamertemplate{theorems}[numbered]  


%## Custom block
\setbeamercolor{block title}{bg=customBlue, fg=white}
\setbeamercolor{block body}{bg=customWhite, fg=customBlack}
\setbeamercolor{block title alerted}{%
    use={block title, alerted text},
    bg=customRed,
    fg=white
}
\setbeamercolor{block body alerted}{%
    use={block title, alerted text},
    bg=customWhite,
    fg=customBlack
}
\AtBeginEnvironment{definition}{%
    \setbeamercolor{block title}{fg=white,bg=customBlackGrey}
    \setbeamercolor{block body}{fg=customBlack, bg=customWhite}
}
\AtBeginEnvironment{theorem}{%
    \setbeamercolor{block title}{fg=white,bg=customBlackGrey}
    \setbeamercolor{block body}{fg=customBlack, bg=customWhite}
}
\AtBeginEnvironment{corollary}{%
    \setbeamercolor{block title}{fg=white,bg=customBlackGrey}
    \setbeamercolor{block body}{fg=customBlack, bg=customWhite}
}
\AtBeginEnvironment{lemma}{%
    \setbeamercolor{block title}{fg=white,bg=customBlackGrey}
    \setbeamercolor{block body}{fg=customBlack, bg=customWhite}
}


%! Useful command
\renewcommand{\Pr}{\text{Pr}}
% $\ast$ \underline{Proof}:
%\checkmark \underline{meaning}:

\title{Braket notation}
\date{\today}
\author{Vaughan Sohn}
% \institute{Centre for Modern Beamer Themes}


\begin{document}
    %#################################### 
    \maketitle
    
    %#################################### 
    \begin{frame}
        \frametitle{Contents}
        \tableofcontents
    \end{frame}

    \begin{section}{Basic operations}
        \begin{frame}
            \frametitle{Basic property}
            \textbf{Vector space}
            \begin{itemize}
                \item $\ket \psi + \ket \phi = \ket \phi + \ket \psi$
                \item $\{\ket \psi + \ket \phi \} + \ket \chi = \ket \psi + \{ \ket \phi  + \ket \chi \}$
                \item $c \{ \ket \psi +  \ket \phi \} = c \ket \psi + c \ket \phi$
                \item $c_1 \ket \psi + c_2 \ket \psi = (c_1 + c_2) \ket \psi$
                \item $\bra \phi ( \ket  \psi + \ket {\psi'}) = \braket{\phi|\psi} + \braket{\phi |\psi'}$
                \item $ (\bra  \psi + \bra {\psi'}) \ket \phi = \braket{\psi|\phi} + \braket{\psi' |\phi}$
            \end{itemize}
            \vspace{0.5cm}
            \textbf{Conjugate transpose}
            \begin{itemize}
                \item $\ket \psi ^\dagger = \bra \psi$
                \item $(\ket \psi + \ket \phi)^\dagger = \bra \psi + \bra \phi$
                \item $(z \ket{\psi})^\dagger = z^\ast \bra{\psi} $
            \end{itemize}
        \end{frame}

    \end{section}

    \begin{section}{Inner product and Outer product}
        \begin{frame}
            \frametitle{Inner product}
            \begin{itemize}
                \item $\braket{\psi | A | \phi}$는 다음 2가지 의미로 생각할 수 있다.
                \begin{itemize}
                    \item $(\ket \psi, A \ket \phi)$: $\ket \psi$와 $A \ket \phi$의 내적
                    \item $(A^\dagger \ket \psi, \ket \phi)$: $A^\dagger \ket \psi$와 $\ket \phi$의 내적
                \end{itemize}
                \item $\bra \phi \{ \ket{\psi_1} + \ket{\psi_2}\} = \braket{\phi|\psi_1} + \braket{\phi|\psi_2}$
                \item $(\braket{\psi|\phi})^\ast = \braket{\phi | \psi}$
            \end{itemize}
        \end{frame}

        \begin{frame}
            \frametitle{Outer product}
            \begin{itemize}
                \item Outer product $\ket w \bra v, (\ket w \in W, \ket v \in V)$의 의미: $V$에서 $W$로 가는 linear operator 
                \item $(\ket w \bra v) \ket v'$는 다음 2가지로 표현할 수 있다.
                \begin{itemize}
                    \item $\ket w \braket{v| v'}$
                    \item $\braket{v| v'} \ket w $
                \end{itemize}
                \item Outer product를 사용하면 어떤 연산자 $A: V \rightarrow W$도 eigenvector를 사용하여 다음과 같이 외적으로 나타낼 수 있다.
                $$ I_V A I_W = \sum_{ij} \ket {v_i} \bra {v_i} A \ket {w_j} \bra {w_j} = \sum_{ij} \underbrace{\braket{v_i|A|w_j}}_{a_{ij}} \ket{v_i} \bra{w_j}$$
                \item 특히 normal operator는 이런 표현을 eigenvector와 eigenvalue로 나타낼 수 있다.
                \begin{itemize}
                    \item 이는 주어진 operator를 대각화 할 수 있음을 의미한다.
                    \item operator function은 이런 표현의 eigenvalue에만 작용한다.
                \end{itemize}
                $$ N = \sum_{i} \lambda_i \ket i \bra i$$
                
            \end{itemize}
        \end{frame}
    \end{section}
    
    \begin{section}{Trace}
        \begin{frame}
            \frametitle{Trace}
                \begin{itemize}
                    \item outer product에서 $\braket{i|A|j}$가 $A_{ij}$를 의미하는 것임을 알았다.
                    \item 따라서 대각성분의 합인 trace는 다음과 같이 표현할 수 있다.
                    $$ \operatorname{tr}(A) = \sum_i \braket{i|A|i}$$
                    \item property:
                    \begin{itemize}
                        \item $\text{tr}(AB) = \text{tr}(BA)$
                        \item $\text{tr}(A+B) = \text{tr}(A) + \text{tr}(B)$
                        \item $\text{tr}(zA) = z \cdot \text{tr}(A)$
                        \item (unitary invariant)
                        $$\text{tr}(UAU^\dagger) = \text{tr}(A)$$
                        \item $\braket{\psi | A |\psi} = \text{tr}[A \ket \psi \bra \psi]$ ($\ast$)
                    \end{itemize}
                \end{itemize}
            
        
        \end{frame}
    \end{section}

    \begin{section}{Tensor product}
        \begin{frame}
            \frametitle{Tensor product}
            \begin{itemize}
                \item For an arbitrary scalar $z$ and vectors $\ket v \in V$ and $\ket w \in W$
                \begin{enumerate}
                    \item $z(\ket v \otimes \ket w) = (z \ket v) \otimes \ket w = \ket v \otimes (z \ket w),$
                    \item $(\ket{v_1} + \ket{v_2}) \otimes \ket w = \ket{v_1} \otimes \ket w + \ket{v_2} \otimes w,$
                    \item $\ket v \otimes (\ket{w_1} + \ket{w_2}) = \ket v \otimes \ket{w_1} + \ket v \otimes \ket{w_2}$
                    \item $\ket v \otimes (\ket w \otimes \ket z) = (\ket v \otimes \ket w) \otimes \ket z$ 
                \end{enumerate}
                \item Tensor product operator 
                $$( A \otimes B ) (\ket v \otimes \ket w) = A \ket v \otimes B \ket w$$
                \item partial trace 
                $$ \begin{aligned} \operatorname{tr}_B (\ket{a_1} \bra{a_2} \otimes \ket{b_1} \bra{b_2}) &= \ket{a_1} \bra{a_2} \operatorname{tr}_B(\ket{b_1} \bra{b_2}) =  \ket{a_1} \bra{a_2} \braket{b_1|b_2} 
                \\ &= \sum_i \braket{i_B|(\ket{a_1} \bra{a_2} \otimes \ket{b_1} \bra{b_2})|i_B} \\  &=  \ket{a_1} \bra{a_2} \otimes\sum_i \braket{i_B|(\ket{b_1} \bra{b_2})|i_B}\\ \end{aligned}$$
            \end{itemize}
        
        \end{frame}
    \end{section}

\end{document}