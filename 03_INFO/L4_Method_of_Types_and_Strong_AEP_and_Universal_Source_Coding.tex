\documentclass[9pt]{beamer}
\usepackage{kotex}
\usepackage{amsfonts,amssymb,amsthm}
\usepackage[dvipsnames]{xcolor}
\usepackage{xcolor}
\usepackage{etoolbox}
\usepackage{braket}
\usepackage{qcircuit}

%## color
\definecolor{customBlack}{HTML}{3B4252}
\definecolor{customBlackGrey}{HTML}{434C5e}
\definecolor{cuatomGrey}{HTML}{4C566A} 
\definecolor{customWhite}{HTML}{ECEFF4} 
\definecolor{customBlue}{HTML}{6082B6}  
\definecolor{customRed}{HTML}{BF616A}
\definecolor{vividauburn}{rgb}{0.58, 0.15, 0.14}


%## Theme & custom
% \usetheme{metropolis}           % Use metropolis theme
% \metroset{block=fill}
\usetheme{moloch} % modern fork of the metropolis theme
\molochset{block=fill}
\setbeamersize{text margin left=5mm, text margin right=5mm}
\setbeamercolor{palette primary}{bg=customBlack}
\setbeamercolor{alerted text}{fg=customRed}
\setbeamercolor{itemize item}{fg=customBlue}
\setbeamercolor{enumerate item}{fg=customBlue}


%## font
\usefonttheme[onlymath]{serif}
% \setbeamerfont{normal text}{size=\small}
% \setbeamerfont{math text}{size=\tiny}


%## Theorem title, numbering
\makeatletter
\setbeamertemplate{theorem begin}
{%
\begin{\inserttheoremblockenv}
{%
\inserttheoremheadfont
\inserttheoremname
\ifx\inserttheoremaddition\@empty\else\ of\ \inserttheoremaddition\fi%
\inserttheorempunctuation
}%
}
\setbeamertemplate{theorem end}{\end{\inserttheoremblockenv}}
\makeatother
\setbeamertemplate{theorems}[numbered]  


%## Custom block
\setbeamercolor{block title}{bg=customBlue, fg=white}
\setbeamercolor{block body}{bg=customWhite, fg=customBlack}
\setbeamercolor{block title alerted}{%
    use={block title, alerted text},
    bg=customRed,
    fg=white
}
\setbeamercolor{block body alerted}{%
    use={block title, alerted text},
    bg=customWhite,
    fg=customBlack
}
\AtBeginEnvironment{definition}{%
    \setbeamercolor{block title}{fg=white,bg=customBlackGrey}
    \setbeamercolor{block body}{fg=customBlack, bg=customWhite}
}
\AtBeginEnvironment{theorem}{%
    \setbeamercolor{block title}{fg=white,bg=customBlackGrey}
    \setbeamercolor{block body}{fg=customBlack, bg=customWhite}
}
\AtBeginEnvironment{corollary}{%
    \setbeamercolor{block title}{fg=white,bg=customBlackGrey}
    \setbeamercolor{block body}{fg=customBlack, bg=customWhite}
}
\AtBeginEnvironment{lemma}{%
    \setbeamercolor{block title}{fg=white,bg=customBlackGrey}
    \setbeamercolor{block body}{fg=customBlack, bg=customWhite}
}


%! Useful command
\renewcommand{\Pr}{\text{Pr}}
% $\ast$ \underline{Proof}:
%\checkmark \underline{meaning}:

\title{4. Method of Types, Strong AEP and Universal Source Coding}
\date{\today}
\author{Vaughan Sohn}
% \institute{Centre for Modern Beamer Themes}


\begin{document}
    %#################################### 
    \maketitle
    
    %#################################### 
    \begin{frame}
        \frametitle{Contents}
        \tableofcontents
    \end{frame}
    


    %#################################### 
    \begin{section}{Types}
        \begin{frame}
            \frametitle{Limitation of weak typical set}
            
        \end{frame}

        \begin{frame}
            \frametitle{Empirical discrete probability distribution}
            
        \end{frame}

        \begin{frame}
            \frametitle{Types}
            
        \end{frame}


        \begin{frame}
            \frametitle{Types : size}
            
        \end{frame}

        \begin{frame}
            \frametitle{Types : probability}
            
        \end{frame}


        \begin{frame}
            \frametitle{Summary}
            
        \end{frame}
    \end{section}




    %#################################### 
    \begin{section}{Method of Types}
        \begin{frame}
            \frametitle{Method of types : size}
            
        \end{frame}

        \begin{frame}
            \frametitle{Method of types : probability}
            
        \end{frame}

        \begin{frame}
            \frametitle{Summary}
            
        \end{frame}
    \end{section}




    %#################################### 
    \begin{section}{Strong Typicality}
        \begin{frame}
            \frametitle{Strong typical set}
            
        \end{frame}

        \begin{frame}
            \frametitle{Strong AEP}
            
        \end{frame}

        \begin{frame}
            \frametitle{Strong typical set size}
            
        \end{frame}

        \begin{frame}
            \frametitle{Comparison between weak typical set}
            
        \end{frame}

        \begin{frame}
            \frametitle{Summary}
            
        \end{frame}
    \end{section}




    %#################################### 
    \begin{section}{Universal Source Coding}
        \begin{frame}
            \frametitle{Universal Source Coding system}
            
            \begin{block}{Idea of universal source coding}
                
            \end{block}
        \end{frame}
        
        \begin{frame}
            \frametitle{Universal Source Coding theorem}
            
        \end{frame}


    \end{section}

    \begin{frame}{References}
        \begin{itemize}
            \item T. M. Cover and J. A. Thomas. Elements of Information Theory, Wiley, 2nd ed., 2006.
            \item Gallager (2008), Principles of Digital Communication, Cambridge University Press.
            \item Lecture notes for EE623: Information Theory (Fall 2024)
        \end{itemize}
        \vspace{6cm}
    \end{frame}


\end{document}