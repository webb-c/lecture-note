\documentclass[9pt]{beamer}
\usepackage{kotex}
\usepackage{amsfonts,amssymb,amsthm}
\usepackage[dvipsnames]{xcolor}
\usepackage{xcolor}
\usepackage{etoolbox}
\usepackage{braket}
%## color
\definecolor{customBlack}{HTML}{3B4252}
\definecolor{customBlackGrey}{HTML}{434C5e}
\definecolor{cuatomGrey}{HTML}{4C566A} 
\definecolor{customWhite}{HTML}{ECEFF4} 
\definecolor{customBlue}{HTML}{6082B6}  
\definecolor{customRed}{HTML}{BF616A}
\definecolor{vividauburn}{rgb}{0.58, 0.15, 0.14}


%## Theme & custom
% \usetheme{metropolis}           % Use metropolis theme
% \metroset{block=fill}
\usetheme{moloch} % modern fork of the metropolis theme
\molochset{block=fill}
\setbeamersize{text margin left=5mm, text margin right=5mm}
\setbeamercolor{palette primary}{bg=customBlack}
\setbeamercolor{alerted text}{fg=customRed}
\setbeamercolor{itemize item}{fg=customBlue}
\setbeamercolor{enumerate item}{fg=customBlue}


%## font
\usefonttheme[onlymath]{serif}
% \setbeamerfont{normal text}{size=\small}
% \setbeamerfont{math text}{size=\tiny}


%## Theorem title, numbering
\makeatletter
\setbeamertemplate{theorem begin}
{%
\begin{\inserttheoremblockenv}
{%
\inserttheoremheadfont
\inserttheoremname
\ifx\inserttheoremaddition\@empty\else\ of\ \inserttheoremaddition\fi%
\inserttheorempunctuation
}%
}
\setbeamertemplate{theorem end}{\end{\inserttheoremblockenv}}
\makeatother
\setbeamertemplate{theorems}[numbered]  


%## Custom block
\setbeamercolor{block title}{bg=customBlue, fg=white}
\setbeamercolor{block body}{bg=customWhite, fg=customBlack}
\setbeamercolor{block title alerted}{%
    use={block title, alerted text},
    bg=customRed,
    fg=white
}
\setbeamercolor{block body alerted}{%
    use={block title, alerted text},
    bg=customWhite,
    fg=customBlack
}
\AtBeginEnvironment{definition}{%
    \setbeamercolor{block title}{fg=white,bg=customBlackGrey}
    \setbeamercolor{block body}{fg=customBlack, bg=customWhite}
}
\AtBeginEnvironment{theorem}{%
    \setbeamercolor{block title}{fg=white,bg=customBlackGrey}
    \setbeamercolor{block body}{fg=customBlack, bg=customWhite}
}
\AtBeginEnvironment{corollary}{%
    \setbeamercolor{block title}{fg=white,bg=customBlackGrey}
    \setbeamercolor{block body}{fg=customBlack, bg=customWhite}
}
\AtBeginEnvironment{lemma}{%
    \setbeamercolor{block title}{fg=white,bg=customBlackGrey}
    \setbeamercolor{block body}{fg=customBlack, bg=customWhite}
}


%! Useful command
% \renewcommand{\Pr}{\text{Pr}}
% $\ast$ \underline{Proof}:
%\checkmark \underline{meaning}:

\title{3. Data Compression, AEP and Lossless Source Coding}
\date{\today}
\author{Vaughan Sohn}
% \institute{Centre for Modern Beamer Themes}


\begin{document}
    %#################################### 
    \maketitle
    
    %#################################### 
    \begin{frame}
        \frametitle{Contents}
        \tableofcontents
    \end{frame}

    
    %#################################### 
    \begin{section}{Source coding}
        \begin{frame}{System}
            \begin{itemize}
                \item 
            \end{itemize}
        \end{frame}

        \begin{frame}{System}
            \begin{definition}[Discrete Memoryless Source: DMS]
                
            \end{definition}
        \end{frame}

        \begin{frame}{Fixed-length source coding}
            \begin{definition}[fixed-length code]
                
            \end{definition}
        \end{frame}

        \begin{frame}{Variable-length source coding}
            \begin{definition}[variable-length code]
                
            \end{definition}
        \end{frame}

    \end{section}

    %#################################### 
    \begin{section}{Decodability and optimality}
        \begin{frame}{Uniquely decodable code}
            \begin{definition}[unique decodability]
                
            \end{definition}
        \end{frame}

        \begin{frame}{Prefix-free code}
            \begin{definition}[prefix-free code]
                
            \end{definition}
        \end{frame}

        \begin{frame}{Condition of optimal code}
            \begin{lemma}
                
            \end{lemma}
            \begin{lemma}
                
            \end{lemma}
            \begin{lemma}
                
            \end{lemma}
            \vspace{0.2cm}
            $\ast$ \underline{Proof}:
        \end{frame}
    \end{section}

    %#################################### 
    \begin{section}{Huffman Code}
        \begin{frame}
            \frametitle{Huffman Code Algorithm}
            \begin{enumerate}
                \item 
            \end{enumerate}
        \end{frame}

        \begin{frame}
            \frametitle{Optimality of Huffman code}
            \begin{lemma}
            
            \end{lemma}
            
            \vspace{0.2cm}
            $\ast$ \underline{Proof}:
        \end{frame}
    \end{section}

    %#################################### 
    \begin{section}{Asmptotic Equipartition Property (AEP)}
        \begin{frame}
            \frametitle{Prerequisites: Weak Law of Large Numbers}
            \begin{theorem}[Weak Law of Large Numbers (WLLN)]
                
            \end{theorem}
            For proof of WLLN, we using markov inequality.
            \begin{theorem}[Markov inequality]
                
            \end{theorem}
            
        
        \end{frame}
        
        \begin{frame}
            \frametitle{Prerequisites: Weak Law of Large Numbers}
            $\ast$ \underline{Proof}:
            
        \end{frame}

        \begin{frame}
            \frametitle{Asymptotic Equipartition Property (AEP)}
            \begin{theorem}[Asymptotic Equipartition Property (AEP)]
                
            \end{theorem}
            \begin{itemize}
                \item 
            \end{itemize}
    
        
        \end{frame}

        \begin{frame}
            \frametitle{Weak typical set}
            \begin{definition}[weak typical set]
                
            \end{definition}
            \begin{itemize}
                \item 
            \end{itemize}
        
        \end{frame}   

        \begin{frame}
            \frametitle{Weak typical set}
            \begin{corollary}
                
            \end{corollary}

        
        \end{frame}  

        \begin{frame}
            \frametitle{Weak typical set}
            \begin{corollary}
                
            \end{corollary}

        
        \end{frame}  

        \begin{frame}
            \frametitle{With exponential approximation }
        
            
        
        \end{frame}
    \end{section}

    %#################################### 
    \begin{section}{Loseless source coding}
        
        \begin{frame}
            \frametitle{Fixed-length block coding}
        
            
        
        \end{frame}
    \end{section}

    \begin{frame}{References}
        \begin{itemize}
            \item T. M. Cover and J. A. Thomas. Elements of Information Theory, Wiley, 2nd ed., 2006.
            \item Gallager (2008), Principles of Digital Communication, Cambridge University Press.
            \item Lecture notes for EE623: Information Theory (Fall 2024)
        \end{itemize}
        \vspace{6cm}
    \end{frame}


\end{document}