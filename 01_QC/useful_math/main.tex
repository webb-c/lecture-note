% Load the kaohandt class (with the default options)
\documentclass[
	%fontsize=10pt, % Base font size
	%twoside=false, % If true, use different layouts for even and odd pages (in particular, if twoside=true, the margin column will be always on the outside)
	%secnumdepth=2, % How deep to number headings. Defaults to 2 (subsections)
	%abstract=true, % Uncomment to print the title of the abstract
]{kaohandt}

% Choose the language
\usepackage[english]{babel} % Load characters and hyphenation
\usepackage[english=british]{csquotes}	% English quotes
\usepackage{kotex}

% Load packages for testing
\usepackage{blindtext}
\usepackage{qcircuit}
%\usepackage{showframe} % Uncomment to show boxes around the text area, margin, header and footer
%\usepackage{showlabels} % Uncomment to output the content of \label commands to the document where they are used

\graphicspath{{images/}{./}} % Paths where images are looked for

% Load mathematical packages for theorems and related environments.
\usepackage{kaotheorems}

% Load the bibliography package
\usepackage{kaobiblio}
\addbibresource{report-template.bib} % Bibliography file

% Load the package for hyperreferences
\usepackage{kaorefs}
\usepackage{braket}

%----------------------------------------------------------------------------------------

\begin{document}

%----------------------------------------------------------------------------------------
%	REPORT INFORMATION
%----------------------------------------------------------------------------------------

\title[Term paper]{Useful things: mathematics}
\author[SV]{Vaughn Sohn}
\date{\today}

%----------------------------------------------------------------------------------------
%	TITLE AND ABSTRACT
%----------------------------------------------------------------------------------------

\maketitle

\margintoc

\begin{abstract}
\noindent
Quantum computing에서 자주 사용되는 수학적 개념이나 정리들
\end{abstract}

% {\noindent\textbf{Keywords:} \LaTeX, Kao, handout, article, report}

\medskip

%----------------------------------------------------------------------------------------
%	MAIN BODY
%----------------------------------------------------------------------------------------

\section{Linear algebra}


\section{Quantum mechanics}
\subsection{Principle}
\begin{definition}[Measurement on density matrix]
    density matrix에 대해서 outcome $m$의 확률은
    $$\operatorname{tr}\left[M_m^{\dagger} M_m \cdot \rho\right],$$
    post-measurement state는 다음과 같다.
    $$\frac{M_m \rho M_m^{\dagger}}{\operatorname{tr}\left[M_m^{\dagger} M_m \cdot \rho\right]}$$
\end{definition}

\subsection{Schrödinger equation}
\begin{theorem}[Schrödinger time evolution equation]
    $$\ket{\psi(t)} = e^{-iHt} \ket{\psi(0)} = U(t) \ket{\psi(0)}$$
\end{theorem}

\section{Complex exponential}
\begin{remark}
    Complex exponential function의 중요한 값
    \begin{itemize}
        \item $e^{i \pi} = e^{2i\pi \times (1/2)} = -1$
        \item $e^{0} = e^{2i\pi \times (0/2)} = 1$
    \end{itemize}
\end{remark}

\begin{remark}
    Complex exponential function과 관련된 몇가지 bound
    \begin{itemize}
        \item $|1 - e^{i\theta} | \ge \frac{2 |\theta|}{\pi}$, for $-\pi \le \theta \le \pi$
        \item $|1 - e^{i\theta} | \le 2$, for any $\theta \in \mathbb R$
    \end{itemize}
\end{remark}

\begin{theorem}[Orthogonality of compex exponential]
    $$\sum_{s=0}^{r-1} e^{(2\pi i s/r)(k-l)} = \begin{cases}
        0, & \text{if } k \ne l\\
        r, & \text{if } k = l
    \end{cases}$$
    special case ($l \leftarrow 0$) : 
    $$\sum_{s=0}^{r-1} e^{(2\pi i s/r)(k)} = \begin{cases}
        0, & \text{if } k \ne 0\\
        r, & \text{if } k = 0
    \end{cases}$$
\end{theorem}

\section{Group theory}

\section{Etc...}
\subsection{Series}
\begin{theorem}[등비 급수]
    일반항이 $a_i = a_1 \cdot r^{n-1}$인 등비급수의 $n$차항 까지의 합은 다음과 같다. 
    $$S_n=a_1 \cdot \frac{r^n-1}{r-1}$$
\end{theorem}

\begin{theorem}[matrix exponential]
    $$e^X=\sum_{k=0}^{\infty} \frac{1}{k!} X^k$$
    where $X^0 = I$.
\end{theorem}

\subsection{Calculus}
\begin{theorem}[Duhamel's principle]
    듀하멜 원리는 비동차(Nonhomogeneous) 선형 미분 방정식의 해를 동차(Homogeneous) 방정식의 해와 적분 형태로 표현하는 방법이다.
    $\tilde U(t), F(t)$가 시간에 의존하고 $H$는 시간에 독립적인 연산자일 때, 비동차 미분 방정식은 다음과 같다.
    $$i \frac{d \tilde U(t)}{d t}=H \tilde U(t)+F(t)$$
    이때 동차 미분 방정식은 다음을 풀어서 쉽게 얻을 수 있다.
    $$i \frac{d \tilde U(t)}{d t}=H \tilde U(t)$$
    Duhamel's principle에 따르면 비동차 미분 방정식의 해를 동차 미분방정식의 해를 사용하여 구할 수 있다. ($U(t)$는 해석적 연산자)\sidenote[][]{Hamiltonian simulation에서도 듀하멜 원리를 적용하기 위해서 주어진 미분방식의 형태를 변경시켰다. 이때 미분방정식의 해가 exponential인 이유는 동차 미분 방정식의 형태가 미분해서 자기자신이 나오는 형태이기 때문이다.}
    $$\tilde U(t) = \tilde U_{\mathrm{sol}}(t)+ -i \int_0^t U(t-\tau) F(\tau) d \tau$$ 
\end{theorem}

\subsection{trigonometric function}
\begin{remark}
    삼각함수에서 자주 쓰이는 몇 가지 값들
    \begin{itemize}
        \item $\sin(\pi/4) = \cos(\pi/4) = \frac{1}{\sqrt 2}$
    \end{itemize}
\end{remark}

\begin{remark}[삼각함수 합 공식]
    \begin{align*}  \sin A + \sin B &= 2\sin\frac{A+B}{2}  \cos \frac{A-B}{2}\\  \sin A - \sin B &= 2\cos\frac{A+B}{2}  \sin \frac{A-B}{2}\\  \cos A + \cos B &= 2\cos\frac{A+B}{2}  \cos \frac{A-B}{2}\\ \cos A - \cos B &= - 2\sin\frac{A+B}{2}  \sin \frac{A-B}{2}\end{align*}
\end{remark}

\subsection{approximation}
\begin{itemize}
    \item $\arccos(\sqrt 1-x) \approx \sqrt x + O(x^{3/2})$
\end{itemize}

%----------------------------------------------------------------------------------------
%	BIBLIOGRAPHY
%----------------------------------------------------------------------------------------

% The bibliography needs to be compiled with biber using your LaTeX editor, or on the command line with 'biber main' from the template directory

\printbibliography[title=Bibliography] % Set the title of the bibliography and print the references

\end{document}