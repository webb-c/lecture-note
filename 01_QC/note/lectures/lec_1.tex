\chapter{Quantum Algorithms}
\lecture{9}{7 Oct. 10:30}{}
\section{Introduction}
이번 챕터에서 우리는 \textit{Quantum Algorithm}에 대해 다루고자한다. Quantum algorithm은 quantum circuit이나 quantum computer에서 구현되는 알고리즘을 지칭한다.
Classical computer가 어려운 문제를 해결하기 위하여 만들어진 것처럼, quantum algorithm에 대해서 공부하고 새로운 방식을 고안하는 것은 quantum computer의 동작방식과 quantum computer의 한계를 분석하기 위한 중요한 과제이다. 
Quantum computer라는 개념이 등장하고 나서부터 지금까지 많은 종류의 quantum algorithm들이 고안되어 왔다. 이번 강의에서 다루고자하는 quantum algorithm은 다음과 같다. 
\begin{itemize}
  \item Elementary quantum algorithms
  \item Hamiltonian simulations
  \item Quantum Fourier transform
  \item Phase estimation
  \item Quantum search algorithm (Grover search algorithm)
  \item Amplitude amplification / estimation algorithms
  \item HHL algorithm
\end{itemize}

\section{Elementary quantum algorithms using quantum parallelism}

Quantum mechanics만의 특징을 이용할 수 있는 quantum computer는 \textit{Quantum parallelism}이라는 특성을 가진다. 이는 quantum computer가 특정 oracle; function $f(x)$에 대하여 동시에 여러개의 입력에 대한 결과를 병렬적으로 얻을 수 있다는 의미이다.
고전적인 개념인 $f(x)$를 quantum computer에서 실행하기 위해서는, quantum computer의 연산단위인 \textit{unitary operator}로 함수를 표현해야하는 필요가 있다.

먼저, 간단한 one-bit boolean function $f : \{0, 1\} \rightarrow \{0, 1\}$를 생각해보자.
직관적으로, 이 함수에 대응되는 operator는 다음과 같이 설계할 수 있다. 이 operator는 $f(0) = 1$이라면, $U_f\ket{0} = \ket{1}$처럼 동작하도록 quantum gate들을 이용하여 구현된다.
\begin{equation*}
  \ket{x} \xrightarrow{U_f} \ket{f(x)}
\end{equation*}
그러나 이러한 방식으로 operator를 설계하게 되면, \textit{non-invertible} 함수 $f(x)$에 대한 operator는 더이상 unitary 조건을 만족하지 못한다.
따라서 이 문제를 해결하기 위하여 control을 수행하는 추가적인 input qubit을 추가하여 unitary operator가 되도록 설계한다.
\begin{definition}[Unitary oracle]\label{def:oracle-gate}
    함수 $f$에 대한 unitary operator $U_f$는 다음과 같이 정의된다.
    \begin{equation*}
      \ket{x}\ket{y} \xrightarrow{U_f} \ket{x}\ket{y \oplus f(x)}
    \end{equation*}
\end{definition}

\newpage 

$\ket x$는 $f$의 입력으로 사용되는 \textbf{oracle qubit}이며, $\ket y$는 1일때는 $f(x)$의 결과를 flip 시키고 0일 때는 $f(x)$의 결과를 반환하는 역할을 수행한다.
그럼 이렇게 설계한 unitary operator에 \textit{superposition state}를 $\ket x$로 제공하면 결과적으로 우리는 다음과 같은 two-qubit state를 얻게 된다.
\begin{equation*}
  \ket{+}\ket{y} \xrightarrow{U_f} \frac{1}{\sqrt 2} (\ket{0}\ket{f(0)} + \ket{1} \ket{f(1)})
\end{equation*}

즉, one-bit boolean function에서 가능한 모든 입력 0, 1에 대한 출력을 $U_f$를 한 번 호출함으로써 얻게 된 것이다!
Classical computer에서 병렬연산은 서로 다른 컴퓨팅 자원을 사용할 뿐, $f$를 여러번 호출해야하는 사실은 변하지 않지만, quantum computer에서의 병렬연산은 실제로 $f$를 한 번 호출하여 모든 연산을 수행할 수 있다.

이렇게 어떤 함수 $f$에 대응되는 unitary operator를 설계하는 것은 n-bit boolean function에 대해서 쉽게 일반화할 수 있다.
$n$-bit boolean function $f: \{0, 1\}^n \rightarrow \{0, 1\}$에 대응되는 unitary operator는 Definition~\ref{def:oracle-gate}을 이용하여 쉽게 설계할 수 있으며, 더 나아가 $n$-qubit에 대한 superposition state를 입력으로 제공하면 다음 결과를 얻게된다.
\begin{equation*}
  \frac{1}{\sqrt{2^n}} \sum_{x \in\{0,1\}^n}|x\rangle|0\rangle \rightarrow \frac{1}{\sqrt{2^n}} \sum_{x \in\{0,1\}^n}|x\rangle|f(x)\rangle
\end{equation*}
이 회로를 이용하면 한 번의 연산만으로 $2^n$개의 입력에 대한 출력을 동시에 얻을 수 있다.
그러나 한 가지 주의해야할 점은, 우리가 결과를 측정하는 순간 여러개의 superposition output은 사라지고 확률에 따라서 단 하나의 결과만을 얻을 수 있다는 사실이다. 
따라서 이러한 quantum parallelism만의 독특한 특징을 잘 활용하여 효과적인 연산을 수행할 수 있도록 알고리즘을 설계하는 것이 중요하다.

이 강의에서는 parallelism의 장점을 활용하는 대표적인 알고리즘들(e.g., Deutsch’s algorithm, Deutsch-Josza algorithm, Simon’s algorithm, and Bernstein-Vazirani algorithm)에 대해 소개한다.

\subsection{Deutsch’s algorithm}
Deutsch’s algorithm은 주어진 one-bit boolean function $f$이 \textbf{balance}인지 \textbf{constant}인지를 판단하는 문제를 해결한다.\footnote{constant: $f(0)=f(1)$, balance: $f(0)\ne f(1)$}
Classical computer가 이 문제를 해결하기 위해서는 두 함숫값 $f(0), f(1)$을 비교하기 위해서 반드시 2번의 함수 호출이 필요하다.
그러나, 지금부터 우리는 quantum computer는 \textbf{단 한번}의 gate call만으로 이 문제를 해결할 수 있음을 보이고자한다.

\begin{enumerate}
  \item input state
  : 우리는 다음과 같은 state를 input으로 제공한다.
  \begin{equation*}
    \ket{\psi} = \ket 0 \ket 1
  \end{equation*}
  \item apply Hadamard gates on both qubits
  : $H$ gate를 가하면, 다음과 같은 상태로 변화한다.
  \begin{equation*}
    \ket{\psi} = \ket + \ket -
  \end{equation*}
  \item apply $U_f$
  : operator를 통과한 state는 다음과 같다.\footnote{$\ket{f(x)} - \ket{1 \oplus f(x)}$는 $f(x)$의 값에 따라서, $\ket{0} - \ket{1}$ ($f(x) = 0$) 또는 $\ket{1} - \ket{0}$ ($f(x) = 1$)이 된다.}
  \begin{align*}
    \ket{\psi} &= U_f\left(\frac{1}{2}\left(\ket{00} + \ket{10} - \ket{01} - \ket{11} \right)\right)\\
                &= \frac{1}{2}\left(\ket{0, f(0)} \ket{1, f(1)} - \ket{0, 1 \oplus f(0)} - \ket{1, 1 \oplus f(1)}\right)\\
                &= \frac{1}{2}\Big(\ket{0}(-1)^{f(0)}(\ket 0 - \ket 1) + \ket{1}(-1)^{f(1)}(\ket 0 - \ket 1) \Big) \\
                &= \frac{1}{2} \sum_{x \in \{0, 1\}}(-1)^{f(x)} \ket x (\ket 0 - \ket 1)
  \end{align*}
  따라서 각 case에 대해 state는 다음의 2가지 모습을 띄게 된다.
  \begin{equation*}
  \ket{\psi} = \begin{cases}
    \pm \ket + \ket - & \text{ if } f(0) = f(1)\\ 
    \pm \ket - \ket - & \text{ if } f(0) \ne f(1)
  \end{cases}
  \end{equation*}
  \item apply again Hadamard gates on oracle qubit
  : 마지막으로 oracle qubit에 $H$ gate를 가하면, $f$의 종류에 따라서 다음과 같은 상태가 된다.
  \begin{equation*}
    \ket{\psi} = \begin{cases}
      \pm \ket 0 \ket - & \text{ if } f(0) = f(1)\\ 
      \pm \ket 1 \ket - & \text{ if } f(0) \ne f(1)
    \end{cases}
  \end{equation*}
\end{enumerate}

따라서, oracle qubit을 측정하면, 100\%의 확률로 0 또는 1의 값을 얻게될 것이며, 그 값에 따라서 우리는 $f$가 balance인지 constant인지를 다음 규칙에 따라서 쉽게 판단할 수 있다. $\Box$
\begin{equation*}
  \ket{\psi} = \begin{cases}
    constant & \text{ if } q_o = 0\\ 
    balance & \text{ if } q_o = 1
  \end{cases}
\end{equation*}

Deutsch’s algorithm은 classical algorithm보다 quantum algorithm 알고리즘이 더 효과적임을 보인 첫 번째 알고리즘이다. 하지만, 그 효과는 단지 2번의 호출을 1번으로 줄일 뿐이다.
따라서 지금부터 더 효과적인 알고리즘들에 대해서 소개하고자한다.

\lecture{10}{14 Oct. 10:30}{}
\subsection{Deutsch-Jozsa algorithm}
Deutsch-Jozsa algorithm은 간단히 말하자면, one-bit boolean function에 대한 문제인 Deutsch-Jozsa algorithm을 $n$-bit boolean function에 대한 문제로 일반화한 것이다.
즉, $n$-bit boolean function $f: \{0, 1\}^n \rightarrow \{0, 1\}$이 \textbf{constant}인지 \textbf{balanced}인지를 판단하는 문제를 해결하는 알고리즘이다.\footnote{여기서 말하는 balanced는 $2^n$개의 input중에서 $2^{n-1}$개의 input에 대한 결과가 0이고, 나머지 $2^{n-1}$개의 input에 대한 결과가 1인 경우를 의미한다.}
마찬가지로 classical computer가 이 문제를 해결하기 위해서는 최대 $2^{n-1}+1$개의 function value를 비교해야하기 때문에 $2^{n-1}+1$번의 함수 호출을 필요로한다. 그러나, 지금부터 이런문제를 해결하려고 할 때도, quantum computer는 \textbf{단 한번}의 gate call만으로 이 문제를 해결할 수 있음을 보일 것이다.

\begin{enumerate}
  \item input state
  : 우리는 다음과 같은 state를 input으로 제공한다. $\{0, 1\}^n$의 가능한 모든 input을 나타내기 위하여, oracle qubit은 $n$개의 qubit으로 구성된다.
  \begin{equation*}
    \ket{\psi} = \ket{0}^{\otimes n} \ket 1
  \end{equation*}
  \item apply Hadamard gates on both qubits: 이때, $\ket{+}^{\otimes n}$은 가능한 모든 $2^n$개의 $n$-bit string들의 superposition이기 때문에, 다음과 같이 표현할 수 있다.
  \begin{align*}
    \ket{\psi} &= \ket{+}^{\otimes n} \ket - \\
              &= \sum_{x \in \{0, 1\}^n} \frac{1}{\sqrt{2^{n}}} \ket x  \frac{1}{\sqrt 2}(\ket 0 - \ket 1)
  \end{align*}
  \item apply $U_f$
  : operator를 통과한 state는 다음과 같다. (Deutsch’s algorithm의 표현을 이용하자)
  \begin{align*}
    \ket{\psi} &= U_f  \frac{1}{\sqrt{2^{n+1}}} \Big(\sum_{x \in \{0, 1\}^n} \ket x  (\ket 0 - \ket 1)\Big)\\
              &= \frac{1}{\sqrt{2^{n+1}}} \sum_{x \in \{0, 1\}^n} (-1)^{f(x)} \ket x  (\ket 0 - \ket 1)
  \end{align*}
  \item apply again Hadamard gates on oracle qubit
  : $H$ gate를 임의의 $n$-qubit computational basis에 가한 결과는 다음과 같다. 이는 single qubit에 대한 동작을 $n$-qubit에 대해 \textit{독립적으로} 적용한 결과이다.
  \begin{equation}
    H^{\otimes n} \ket x = \frac{1}{\sqrt{2^n}} \sum_{z \in \{0, 1\}^n} (-1)^{x \cdot z} \ket z 
  \end{equation}\label{eq:tensor-H}
  이를 이용하여 $H$ gate를 적용한 state를 표현하면, 다음과 같다.
  \begin{equation*}
    \ket \psi  = \frac{1}{{2^n}} \sum_{x, z \in \{0, 1\}^n} (-1)^{f(x) + x \cdot z} \ket z \ket{-}
  \end{equation*}
  
  만약 $f(x)$가 constant function이라면, $\forall x$에 대해서 $f(x)$의 값은 항상 동일하기 떄문에, $(-1)^{f(x)}$의 값이 $+1$, 또는 $-1$이라는 constant가 되어 다음과 같이 나타낼 수 있다.
  \begin{align*}
  \left|\psi\right\rangle= \pm \frac{1}{{2^{n}}} \sum_{x, z \in\{0,1\}^n} {(-1)^{x\cdot z}|z\rangle} \ket -
  \end{align*}
  $z = 0^n$인 경우를 가정해보자. 이는 $x$가 어떤 값이 되던지간에 $x\cdot z$ \footnote{ $x \cdot z = x_1 z_1 + x_2 z_2 + \cdots + x_n z_n \mod 2$ }의 값이 0이 되기 떄문에, 다음과 같이 표현할 수 있게 된다.
  $\ket{0}^{\otimes n}$의 amplitude의 square norm이 1이기 때문에, 100\% 확률로 $0^n$을 측정할 수 있다.
  \begin{align*}
    \pm \frac{1}{{2^n}} \sum_{x \in\{0,1\}^n} (-1)^{x\cdot 0^n} |0\rangle^{\otimes n} \ket - = \pm \frac{1}{{2^n}}{2^n} \ket{0}^{\otimes n} \ket - = \boxed{\pm \ket{0}^{\otimes n} \ket -} 
  \end{align*}
  반면, $f(x)$가 balance function이라면, $f(x)$의 값이 $0$인 경우와 $1$인 경우가 정확히 1/2씩 나타나기 때문에, 각 항들이 소거되면서 $z=0^n$에 대한 amplitude가 0이 된다.
  \begin{align*}
    \Big(\frac{1}{{2^n}} 2^{n-1} |0\rangle^{\otimes n} \ket - \Big) + \Big( - \frac{1}{{2^n}} 2^{n-1} |0\rangle^{\otimes n} \ket - \Big) = \boxed{0 \ket{0}^{\otimes n} \ket -} 
  \end{align*}
\end{enumerate}
따라서, oracle qubit을 측정한 결과가 $0^n$인지 확인하여, $f$가 constant인지 balanced인지를 알 수 있다. $\Box$
  \begin{equation*}
    \ket{\psi} = \begin{cases}
      constant & \text{ if } q_o = 0^n\\ 
      balance & \text{ if } q_o \ne 0^n
    \end{cases}
  \end{equation*}

\section{Hamiltonian simulations}
Hamiltonian simulation은 quantum computer가 고안된 핵심적인 이유 중 하나이다. 리처드 파인만의 말을 인용하자면 자연, 그중에서도 특히 미시세계는 고전역학을 따르지 않기때문에, quantum mechanical을 따르는 simulation이 필요하다.
\begin{note}
  \textit{Nature isn’t classical, dammit, and if you want to make a simulation of nature, you’d better make it quantum mechanical, and by golly it’s a wonderful problem, because it doesn’t look so easy.}
\end{note}

Hamiltonian simulation은 간단히 말해 quantum state의 time evolution을 구하는 것이다. Schrödinger equation에 의하면, initial state $\ket{\psi(0)}$의 시간 $t$에 대한 time evolution은 다음과 같이 주어진다.
\begin{equation*}
  |\psi(t)\rangle=e^{-i H t}|\psi(0)\rangle
\end{equation*}

이 방정식을 곧바로 해결하여 문제를 풀 수도 있지만, Hamiltonian matrix의 크기가 exponential하게 증가하기 때문에 이를 classical algorithm으로 효과적으로 해결하기는 어렵다.

\subsection{Solution of Hamiltonian simulations}
우리는 \hyperref[def:k-local-Hamiltonian]{k-local Hamiltonian}을 이용하여 Hamiltonian simulation 문제를 해결하고자 한다.
\begin{definition}\label{def:k-local-Hamiltonian}
  $k$-local\footnote{여기서 local은 geometrically local이 아니라 단순히 system의 '개수'를 나타내기 위해 사용된다.} Hamiltonian은 주어진 $H$가 $n$-qubit system에서 \textbf{최대 $k$개의 system}에 대해서만 동작하는 $H_i$들의 합으로 표현되는 Hamiltonian이다. (이때 $L$은 $n$에 대해 polynomial이다.)
  $$ H=\sum_{i=1}^L H_i $$
\end{definition}

\begin{eg}
  예를 들어, 다음과 같이 정의되는 Hamiltonian은 2개의 system (2, 4)에 대해서만 동작한다.
  $$ H_i = I_1 \otimes Z_2 \otimes I_3 \otimes X_4 \otimes I_{\perp} $$
\end{eg}

\newpage 
Hamiltonian이 k-local Hamiltonian이라고 가정할 때, 이를 이용하여 어떻게 문제를 해결할 수 있을까?
그 아이디어는 단순하다. $e^{-i H t}$를 계산하는 것은 어렵지만, $e^{-i H_i t}$는 최대 $k << n$ subsystem에만 작용하기에 더 단순하며, 그 단순성 덕분에 quantum circuit을 사용하여 시뮬레이션 하기 쉽다.

만약, $[H_j, H_k] = 0$\footnote{$[H_j, H_k] = H_jH_k - H_kH_j$}이 성립한다면 $e^{-iHt} =e^{-i \sum H_j t} = \prod e^{-i H_j t}$\footnote{지수함수의 성질이 성립하기 위해서 필요한 조건. 실수들은 항상 commutative하기 때문에 이 성질이 자명하게 정립되었던 것}가 성립하기 때문에, 각각의 $H_i$에 대한 time evolution을 독립적으로 계산한 결과를 곱하여 전체 time evolution을 쉽게 구할 수 있다.
그러나 일반적으로 $[H_j, H_k] = 0$은 성립하지 않기 떄문에, 우리는 \hyperref[thm:trotter-formula]{Trotter formula}를 이용한다.

\begin{theorem}[Trotter formula]\label{thm:trotter-formula}
  Hermitian operator $H_j, H_k$에 대하여, 어떤 $t$에 대해서도 다음이 성립한다.\footnote{직관적으로, 주어진 time $t$ 대신에, 매우 짧은 시간간격인 $t/m$에 대한 time evolution을 $m$번 반복하여 $t$에 대한 time evolution을 근사할 수 있다는 것을 의미한다.}
  $$ e^{i (H_j + H_k) t} = \lim_{m \to \infty} \left( e^{i H_j t/m} e^{i H_k t/m} \right)^m $$ 
  이는 $H_j, H_k$가 not-commute라 하더라도 항상 성립한다.
\end{theorem}

\vspace{1em} 

따라서 $L=2$인 k-local Hamiltonian $H = H_1 + H_2$에 대해, Hamiltonian simulation을 수행하는 propagator $\tilde U (t) = e^{-i H_1t} e^{-i H_2 t}$를 시간에 대해서 미분하면, 다음과 같이 전개할 수 있다.
(with initial condition $\tilde U(0) = I$)
\begin{itemize}
  \item Eq.~\eqref{eq:expand-2}: $H_1, H_2$는 matrix라서 순서를 바꿀 수 없기에, 다른 항을 더하여 항을 정리하고 다시 뺀다.
  \item Eq.~\eqref{eq:expand-3}: commutator를 사용하여 항을 정리한다.
\end{itemize}
\begin{align}
  i \frac{d \tilde{U}(t)}{d t} & =H_1 e^{-i H_1 t} e^{-i H_2 t}+e^{-i H_1 t} H_2 e^{-i H_2 t} \label{eq:expand-1} \\ 
  & =\left(H_1+H_2\right) e^{-i H_1 t} e^{-i H_2 t}+e^{-i H_1 t} H_2 e^{-i H_2 t}-H_2 e^{-i H_1 t} e^{-i H_2 t} \label{eq:expand-2} \\ 
  & = H e^{-i H_1 t} e^{-i H_2 t}+\left[e^{-i H_1 t}, H_2\right] e^{-i H_2 t} \label{eq:expand-3} \\ 
  & = H \tilde U(t)+\left[e^{-i H_1 t}, H_2\right] e^{-i H_2 t} \label{eq:expand-4}
\end{align}

미분 방정식에 대한 Duhamel's formula \footnote{See \url{https://en.wikipedia.org/wiki/Duhamel\%27s_principle}}를 이용하면, $\tilde U(t)$를 다음과 같이 구할 수 있다. 
\begin{equation}
  \tilde{U}(t)=U(t)-i \int_0^t e^{-i H(t-s)}\left[e^{-i H_1 s}, H_2\right] e^{-i H_2 s} ds \label{eq:tilde-U}
\end{equation}

\subsection{Performance}
그렇다면, 이제 실제 Hamiltonian에 대한 time evolution $U(t) = e^{-iHt}$와 Trotter formula를 이용하여 근사한 $\tilde U(t)$의 error가 어떻게 bound 되는지 분석해보자. Eq.~\eqref{eq:tilde-U}의 항을 이항시키면, 두 operator의 차이에 대한 norm의 upper bound를 구할 수 있다.
\footnote{$e^{-iH_it}$는 unitary operator이기 때문에 norm을 변화시키지 않기 때문에 무시해도 된다.}
\begin{equation}
  \|\tilde{U}(t)-U(t)\|  = \left\|-i \int_0^t e^{-i H(t-s)}\left[e^{-i H_1 s}, H_2\right] e^{-i H_2 s} d s \right\| \leq \int_0^t\left\|\left[e^{-i H_1 s}, H_2\right]\right\| ds \label{eq:distance}
\end{equation}
이때, norm은 다음과 같이 정의된다.

\begin{definition}
  $$  \|A\| \equiv \max _{|\psi\rangle} \| A|\psi\rangle \|_2=\max _{|v\rangle \ne 0} \frac{\| A|v \rangle \|_2}{\||v\rangle \|_2} $$
\end{definition}

이제 이 norm이 적절한 error rate $\epsilon$에 의하여 bound됨을 보임으로써, 이 Hamiltonian simulation을 실제로 활용할 수 있을지 분석할 수 있다. 

\newpage 
이를 위하여 다음과 같은 새로운 연산자를 가정하자.
\begin{equation}
  G(t) \triangleq [e^{-iH_1t}, H_2] e^{iH_1t} = e^{-i H_1t} H_2 e^{iH_1t} - H_2, \text{ with } G(0) = 0 \label{eq:def-G}
\end{equation}
이 연산자에 대하여 $\tilde U(t)$처럼 시간에 대한 도함수를 구하면 다음과 같고,
\begin{equation*}
  i \frac{d}{d t} G(t)=e^{-i H_1 t}\left[H_1, H_2\right] e^{i H_1 t}
\end{equation*}
양변을 $t$에 대해서 적분하면 다음과 같다. 
\begin{equation*}
  G(t)=G(0) - i \int_{0}^{t} e^{-i H_1 s}\left[H_1, H_2\right] e^{i H_1 s} ds 
\end{equation*}
norm을 취하고 \textit{triangle inequality}를 이용하면, $\|G(t)\|$에 대한 upper bound를 얻는다.
\begin{equation}
  \|G(t) \|=  \left\| -i \int^t_0 e^{-i H_1 s}\left[H_1, H_2\right] e^{i H_1 s} ds \right\| \le \int^t_0 \|\left[H_1, H_2\right]\| ds = t \|\left[H_1, H_2\right]\| \label{eq:G-bound}
\end{equation}
이때, $G(t)$의 norm은 자기자신의 정의 \eqref{eq:def-G}에 의하여 다음 관계가 성립하게 된다.
\begin{equation*}
  \left\|\left[e^{-i H_1 t}, H_2\right]\right\|=\|G(t)\| \leq t\left\|\left[H_1, H_2\right]\right\| 
\end{equation*}
따라서 이를 Eq.~\eqref{eq:distance}에 대입하면, 다음을 얻는다.
\begin{equation*}
  \|\tilde{U}(t)-U(t)\|  \leq \int_0^t \underbrace{\left\|\left[e^{-i H_1 s}, H_2\right]\right\|}_{s\|[H_1, H_2]\|} d s  \leq \int^t_0 s\left\|\left[H_1, H_2\right]\right\|  ds  
\end{equation*}
$\|[H_1, H_2]\|$는 $s$에 관계없는 상수이기 때문에, 다음과 같이 정리할 수 있으며, 
\begin{equation*}
  \|\tilde{U}(t)-U(t)\| \leq \frac{t^2}{2} \left\|\left[H_1, H_2\right]\right\|
\end{equation*}
commutator를 전개한 뒤, triangle inequality와 norm의 성질\footnote{$\|AB\| \le \|A\| \|B\|$}을 이용하면 다음과 같이 전개할 수 있다.
\begin{align*}
  \|\tilde{U}(t)-U(t)\| &\leq \frac{t^2}{2} \left\|\left[H_1, H_2\right]\right\|  = \frac{t^2}{2} \left\|H_1H_2 + (-H_2H_1)\right\| \leq  \frac{t^2}{2} (\|H_1H_2\| + \| H_2H_1\|) \\
                        &\leq \frac{t^2}{2} 2\|H_1\| \|H_2\| \le t^2 \max\{\|H_1\|, \|H_2\|\}^2
\end{align*}
정리하면, 다음과 같다.
\begin{equation}
  \|\tilde{U}(t)-U(t)\| = \|{U}(t)- \tilde U(t)\|  \le t^2 \max\{\|H_1\|,\|H_2\|\}^2  \label{eq:bound}
\end{equation}

\vspace{1em}

따라서 시간간격 $\Delta t$에 대한 error는 다음과 같이 bound된다.
\begin{equation}
 \left\|e^{-i H \Delta t}-\left(e^{-i H_1  \Delta  t} e^{-i H_2 \Delta  t}\right)\right\| \le (\Delta t)^2 \max\{\|H_1\|,\|H_2\|\}^2  \label{eq:bound}
\end{equation}
\hyperref[thm:trotter-formula]{Trotter formula}에 의하여 시간간격 $\Delta t = t/m$에 대하여 $m$단계 근사의 오차는 $\Delta t$에 대한 단일 단계 오차의 누적으로 생각할 수 있기 때문에, 다음을 얻는다.
\begin{equation*}
  \left\|e^{-i H t}-\left(e^{-i H_1  \Delta t} e^{-i H_2 \Delta t}\right)^m\right\| \leq m \frac{t^2}{m^2} \max \left\{\left\|H_1\right\|,\left\|H_2\right\|\right\}^2=\frac{t^2}{m} \max \left\{\left\|H_1\right\|,\left\|H_2\right\|\right\}^2
\end{equation*}
이 bound를 사용하면, 우리가 원하는 target error rate $\epsilon$을 달성하기 위해서 필요한 $m$의 값을 $m = O(t^2 \epsilon^{-1})$으로 결정할 수 있다.\footnote{$\max \{\|H_1 \|, \|H_2 \|\}^2$ (i.e., $\|H_i\|$)의 값은 상수라서 무시하였다.}

이 과정을 더 많은 term을 가지는 k-local Hamiltonian에 대해서 일반화할 수 있다.
\begin{equation*}
  \left\|e^{-i \sum_{i=1}^L H_i \Delta t}-\prod_{j=1}^L e^{-i H_i \Delta t}\right\|=O\left(\frac{t^2}{L} \sum_{i<j}\left\|\left[H_i, H_j\right]\right\|\right) .
\end{equation*}
\newpage
$\Delta t = t/m$로 가정하면 다음을 얻는다.
\begin{equation*}
\left\|e^{-i \sum_{i=1}^L H_i t}-\left(\prod_{j=1}^L e^{-i H_i \Delta t}\right)^m\right\|=O\left(\frac{m \Delta t^2}{L} \sum_{i<j}\left\|\left[H_i, H_j\right]\right\|\right)=O\left(\frac{t^2}{m L} \sum_{i<j}\left\|\left[H_i, H_j\right]\right\|\right)
\end{equation*}
이떄, $\|H_i\| = O(1)$이므로 $\sum \|[H_i, H_j]\| = L^2$이 되어 다음과 같이 bound된다.
\begin{equation*}
\left\|e^{-i \sum_{i=1}^L H_i t}-\left(\prod_{j=1}^L e^{-i H_i \Delta t}\right)^m\right\|=O\left(\frac{L t^2}{m}\right) .
\end{equation*}
$m=O(Lt^2 \epsilon^{-1})$로 설정하게되면 우리는 항상 target error rate $\epsilon$을 upper bound로 가지는 근사 operator $\tilde U(t)$를 구성할 수 있고 이를 이용하여 simulation 할 수 있다.
따라서, 이 solution은 주어진 $k$-local Hamiltonian에 대하여, time $t$에 대한 quadratic overhead (i.e., error)를 가지고 simulation을 할 수 있음을 보여준다. 즉, simulation time이 증가하더라도 error rate은 polynomial하게 증가하게 된다. $\Box$

\vspace{1em}

반면, 2nd-Trotter formula를 이용하여 표현할 수도 있다. 
\begin{equation*}
  e^{i(A+B) \Delta t}=e^{i A \Delta t / 2} e^{i B \Delta t} e^{i A \Delta t / 2}+O\left((\Delta t)^3\right) 
  \end{equation*}
이를 이용하면, time $t$에 대하여 error는 다음과 같이 표현된다.
\begin{equation*}
  \left\|e^{-i(A+B) t}-\prod_{i=1}^m e^{-i A \Delta t / 2} e^{-i B \Delta t} e^{-i A \Delta t / 2}\right\|=O\left(m(\Delta t)^3\right)=O\left(t^3 / m^2\right)
  \end{equation*}
즉, $m = O(t^{3/2}\epsilon^{1/2})$으로 설정하면, target error rate $\epsilon$을 달성할 수 있으며 이는 1st-Trotter formula를 사용한 simulation보다 더 효율적이다.\footnote{time $t$에 대한 3/2 overhead}
2nd-Trotter 방법을 $L$개의 term을 갖는 Hamiltonian에 대하여 일반화하면 $m$은 다음과 같다.
\begin{equation*}
  m=O\left(\frac{\left(\sum_{j=1}^L\left\|H_j\right\| t\right)^{3 / 2}}{\epsilon^{1 / 2}}\right)
  \end{equation*}
더 나아가 pth order Trotter formula를 사용하면, $m$은 다음과 같다.
\begin{equation*}
  m=O\left(\frac{\left(\sum_{j=1}^L\left\|H_j\right\| t\right)^{1+1 / p}}{\epsilon^{1 / p}}\right)
  \end{equation*}

Hamiltonian simulation을 위하여 다양한 연구들이 현재까지도 진행되고 있으며, 대표적인 알고리즘들은 다음을 참고하라.
\begin{itemize}
  \item Higher-order product formula \cite{childs2021theory}
  \item Linear combination of unitary (LCU) \cite{berry2015hamiltonian}
  \item Quantum signal processing (\textit{optimal}) \cite{haah2019product}
\end{itemize}

\lecture{11}{16 Oct. 10:30}{}
\section{Quantum Fourier transform}
\subsection{Quantum Fourier transform}
Quantum Fourier transform은 Shor algorithm이나 HHL algorithm과 같은 다양한 quantum algorithm에서 사용되는 중요한 알고리즘이다.
QFT에 대해서 소개하기에 앞서, 먼저 classical Fourier transform에 대해서 간단히 알아보자.

\begin{definition}[Discrete Fourier transform]
  Discrete Fourier transform은 vector $\mathbf x \in \mathbb C^N$을 입력으로 받아서, 다음과 같이 정의되는 연산을 수행하여 output vector $\mathbf y \in \mathbb C^N$으로 변환하는 과정이다.\footnote{$\mathbf x = x_0\cdots x_{N-1}$, $\mathbf y = y_0\cdots y_{N-1}$ }
  $$ y_k \equiv \frac{1}{\sqrt{N}} \sum_{j=0}^{N-1} x_j e^{2 \pi i j k / N} $$
\end{definition}

\begin{definition}[Quantum Fourier transform]\label{def:QFT}
  Quantum Fourier transform은 DFT와 유사하게, quantum state vector $\ket x = \ket{x_0\cdots x_{N-1}}$을 입력으로 받아서, output quantum state vector $\ket y  =\ket{ y_0\cdots y_{N-1}}$으로 변환하는 과정이다. 
  단, DFT와는 다르게 \textbf{computational basis} $\{\ket 0, \cdots, \ket{N-1}\}$에 대한 변환만이 정의되어 있다.
  $$
    |j\rangle \rightarrow \frac{1}{\sqrt{N}} \sum_{k=0}^{N-1} e^{2 \pi i j k / N}|k\rangle .
  $$
  따라서 임의의 state vector $\ket x$의 transform은 변환된 basis vector $\ket{k}$들의 linear combination으로 표현하게 된다.
  $$
    \sum_{j=0}^{N-1} x_j|j\rangle \rightarrow \sum_{j=0}^{N-1} x_j \frac{1}{\sqrt{N}} \sum_{k=0}^{N-1} e^{2 \pi i j k / N}|k\rangle=\sum_{k=0}^{N-1} y_k|k\rangle,
  $$
  이때, $y_k$는 다음과 같다.\footnote{DFT와 똑같다.}
  $$
    y_k=\frac{1}{\sqrt{N}} \sum_{j=0}^{N-1} x_j e^{2 \pi i j k / N}.
  $$
\end{definition}

\vspace{1em}
즉, QFT가 하는일은 basis $\{\ket k\}$에 대한 linear combination으로 표현된 벡터 $\ket x$를 다른 basis $\{\ket j\}$에 대한 linear combination으로 나타내는 \textit{basis transform}이다.\footnote{$\{\ket k\}$ basis에서의 amplitude는 $x_i$이며, $\{\ket j\}$ basis에서의 amplitude는 $y_i$이다.}

\vspace{1.5em}
이제 QFT가 어떻게 구현되는지 알아보자. $\ket j$에 대한 변환을 수행하는 QFT의 circuit은 다음과 같다.
\begin{figure}[h]
  \[
  \begin{array}{c}
  \Qcircuit @C=0.6em @R=0.6em {
      \lstick{\ket{j_1}} & \gate{H} & \gate{R_2}  &\qw & \cdots &  & \gate{R_{n-1}}&\gate{R_n}& \qw      &\qw &\qw    &\qw           &\qw     &\qw       &\qw&\qw &\qw   &\qw      &\qw        &\qw   &\qw    &\rstick{\frac{1}{\sqrt 2}\ket 0 + e^{2\pi i \ 0.j_1 \cdots j_n} \ket 1}\\
      \lstick{\ket{j_2}} & \qw      & \ctrl{-1}   &\qw & \cdots &  & \qw           &\qw       & \gate{H} &\qw & \cdots &  & \gate{R_{n-2}}&\gate{R_{n-1}}&\qw & \cdots &  &\qw      &\qw        &\qw   &\qw    &\rstick{\frac{1}{\sqrt 2}\ket 0 + e^{2\pi i \ 0.j_2 \cdots j_n} \ket 1}\\
      \vdots             &          &             &    & & &           &               &         &          & & &       &              &               & &       &         &           &        & & &  &\vdots  \\
      \lstick{\ket{j_{n-1}}} & \qw  & \qw         &\qw &\qw  &\qw   & \ctrl{-3}     &\qw       &\qw      &\qw &\qw    &\qw    &\ctrl{-2}     &\qw            &\qw& \cdots &  &\gate{H} &\gate{R_2} &\qw   &\qw    &\rstick{\frac{1}{\sqrt 2}\ket 0 + e^{2\pi i \ 0.j_{n-1} j_n} \ket 1}\\
      \lstick{\ket{j_n}} & \qw      & \qw         &\qw &\qw &\qw   & \qw           &\ctrl{-4} &\qw       &\qw &\qw   &\qw       &\qw    &\ctrl{-3}      &\qw& \cdots &  &\qw      &\ctrl{-1}  &\gate{H} &\qw  &\rstick{\frac{1}{\sqrt 2}\ket 0 + e^{2\pi i \ 0.j_n} \ket 1}
  }
  \end{array}
  \]
  \caption{QFT circuit} \label{fig:QFT-circuit}
\end{figure}

QFT가 $n$-qubit system에 대해서 computational basis를 변환시킬 때, 우리는 computational basis를 나타내기 위하여 binary representation을 도입하고자한다.
어떤 $\ket j \in \{\ket 0, \cdots, \ket{2^n-1}\}$에 대한 binary representation은 다음과 같다. ($N=2^n$)
\begin{equation*}
  j=j_1 j_2 \ldots j_n=j_1 2^{n-1}+\cdots+j_n 2^0=\sum_{k=1}^n j_k 2^{n-k} .
  \end{equation*}
또한, 소수도 다음과 같은 binary representation으로 표현할 수 있다.
\begin{equation*}
  0 . j_l j_{l+1} \cdots j_m=j_l / 2+j_{l+1} / 2^2 +\cdots+j_m / 2^{m-l+1}=\sum_{k=l}^m j_k / 2^{k-l+1}
  \end{equation*}

\newpage
Binary representation을 이용하면 QFT의 연산을 다음과 같이 분석할 수 있다.
\begin{itemize}
  \item Eq.~\eqref{eq:QFT-1}: Definition~\ref{def:QFT}에 따라, $|j\rangle$에 대한 QFT는 다음과 같이 표현된다.
  \item Eq.~\eqref{eq:QFT-2}: $k = \sum k_l 2^{n-l}$이므로 $k/2^n = \sum k_l 2^{-l}$이다.
  \item Eq.~\eqref{eq:QFT-3}: $e^{a+b} = e^a e^b$, 그리고 $\ket k$가 $n$-qubit에 대해 tensor product로 표현됨을 이용한다.
  \item Eq.~\eqref{eq:QFT-4}: 표현 단순화. ($\sum_{k_1, k_2, \cdots, k_n \in \{0, 1\}}$ 을 $\sum_{k_l \in \{0, 1\}}$로 표현)
  \item Eq.~\eqref{eq:QFT-5}: (1) $k_l=0$, then $e^{2\pi ij k_l 2^{-l}} = e^0$. (2) $k_l=1$, then $e^{2\pi ij k_l 2^{-l}} = e^{2\pi i j 2^{-l}}$
  \item Eq.~\eqref{eq:QFT-6}: 모든 tensor product들을 전개한뒤 fraction을 binary representation으로 표현한다.
\end{itemize}
\begin{align}
  |j\rangle & \rightarrow \frac{1}{2^{n / 2}} \sum_{k=0}^{2^n-1} e^{2 \pi i j k / 2^n}|k\rangle \label{eq:QFT-1} \\
  & =\frac{1}{2^{n / 2}} \sum_{k_1=0}^1 \cdots \sum_{k_n=0}^1 e^{2 \pi i j\left(\sum_{l=1}^n k_l 2^{-l}\right)}\left|k_1, \ldots, k_n\right\rangle \label{eq:QFT-2} \\
  & =\frac{1}{2^{n / 2}} \sum_{k_1=0}^1 \cdots \sum_{k_n=0}^1 \bigotimes_{l=1}^n e^{2 \pi i j k_l 2^{-l}}\left|k_l\right\rangle  \label{eq:QFT-3} \\
  & =\frac{1}{2^{n / 2}} \bigotimes_{l=1}^n \sum_{k_l=0}^1 e^{2 \pi i j k_l 2^{-l}}\left|k_l\right\rangle \label{eq:QFT-4} \\
  & =\frac{1}{2^{n / 2}} \bigotimes_{l=1}^n\left[|0\rangle+e^{2 \pi i j 2^{-l}}|1\rangle\right]  \label{eq:QFT-5} \\
  & =\frac{\left(|0\rangle+e^{2 \pi i 0 . j_n}|1\rangle\right)\left(|0\rangle+e^{2 \pi i 0 . j_{n-1} j_n}|1\rangle\right) \cdots\left(|0\rangle+e^{2 \pi i 0 . j_1 j_2 \ldots j_n}|1\rangle\right)}{2^{n / 2}} \label{eq:QFT-6}
\end{align}
즉, $\ket j$의 각 qubit $\ket{j_i}$에 대해 독립적으로 특정 gate $U$를 적용하여 $\ket 0 + e^{2\pi i 0.j_i \cdots j_n}$가 되도록 quantum circuit을 설계하면, 그 결과가 QFT에 해당한다는 사실을 알아냈다.
\begin{equation*}
  U \ket {j_i} \rightarrow \ket 0 + e^{2\pi i 0.j_i \cdots j_n} \ket 1
\end{equation*}

\vspace{1em}
본격적으로 quantum circuit을 만들기 위해서 rotation operator $R_k$를 다음과 같이 정의하자. 이렇게 정의한 operator는 $\ket 0$에 대해서는 아무것도 수행하지 않지만, $\ket 1$에 대해서는 phase $e^{2\pi i / 2^k}$를 적용한다.
\begin{equation*}
  R_k \equiv\left(\begin{array}{cc}
  1 & 0 \\
  0 & e^{2 \pi i / 2^k}
  \end{array}\right)
\end{equation*}
Fig.~\ref{fig:QFT-circuit}의 연산을 단계별로 따라가보자.
\begin{enumerate}
  \item input state : 다음과 같은 input state로 시작한다. 이는 computational basis state중 하나이다.
  \begin{equation*}
    \ket \psi = \ket{j_1,\cdots, j_n}
  \end{equation*}
  \item apply Hadamard gate on the first qubit : 첫 번째 qubit; $\ket{j_1}$에 $H$를 적용하면, 다음을 얻는다.
  \footnote{이때, $e^{i\pi} = {\color{gray}e^{2i\pi \frac{1}{2}}} = -1$ 그리고 $e^{0} = {\color{gray}e^{2i\pi \frac{0}{2}}} = 1$라는 사실을 이용한다.}
  \begin{equation*}
    \ket \psi = \frac{1}{\sqrt 2} (\ket{0} + (-1)^{j_1}\ket{1})\ket{j_2, \cdots, j_n} = \frac{1}{\sqrt 2} \left( \ket 0 + e^{2\pi i 0.j_1} \ket 1 \right) \ket{j_2, \cdots, j_n}
  \end{equation*}
  \item apply controlled-$R_2$ gate with the first qubit as the \textit{target} and the second qubit as \textit{control}.
  \begin{align*}
    \ket \psi &= \begin{cases} 
        \frac{1}{\sqrt 2} \left( \ket 0 + e^{2\pi i 0.j_1} \ket 1 \right) \ket{j_2, \cdots, j_n} & \text{if } j_2 = 0, \\ 
        \frac{1}{\sqrt 2} \left( \ket 0 + e^{2\pi i 0.01} e^{2\pi i 0.j_1} \ket 1 \right) \ket{j_2, \cdots, j_n} & \text{if } j_2 = 1 
    \end{cases}\\ 
              &= \frac{1}{\sqrt 2} \left( \ket 0 + e^{2\pi i 0.j_1j_2} \ket 1 \right) \ket{j_2, \cdots, j_n}.
\end{align*}
\newpage
  \item apply controlled-$R_k$ gate consequently : 3번의 과정을 control qubit을 하나씩 증가시키면서 반복한다. 이때, control qubit의 순서가 $i$번째라면, $R_i$ gate를 적용해야한다.
  \begin{equation*}
    \ket \psi = \frac{1}{\sqrt 2} \left( \ket 0 + e^{2\pi i 0.j_1j_2 \cdots j_n} \ket 1 \right) \ket{j_2, \cdots, j_n} 
  \end{equation*}
  \item apply controlled-$R_k$ gate consequently with other \textit{target state}  : 3-4번 과정을 다른 control qubit에 대해서 반복한다. 
  
  예를 들어, second qubit을 target qubit으로서 가정하면, 다음과 같은 state를 얻게된다.
  \begin{equation*}
    \ket \psi = \frac{1}{\sqrt 2^2} \left( \ket 0 + e^{2\pi i 0.j_1j_2 \cdots j_n} \ket 1 \right) \left( \ket 0 + e^{2\pi i 0.j_2j_3 \cdots j_n} \ket 1 \right) \ket{j_3, \cdots, j_n} 
  \end{equation*}
  일반화하면, $j$번째 qubit에 대해, $k(j < k)$번쨰 qubit을 control qubit으로 controlled-$R_{k-j}$ gate를 차례대로 적용하는 과정을 반복한다. 
  \item $n$번쨰 qubit까지 이 과정을 수행하면 최종적으로 다음 state를 얻게된다.
  \begin{equation*}
    \ket \psi = \frac{1}{\sqrt 2^n} \left( \ket 0 + e^{2\pi i 0.j_1j_2 \cdots j_n} \ket 1 \right) \left( \ket 0 + e^{2\pi i 0.j_2j_3 \cdots j_n} \ket 1 \right) \cdot \left( \ket 0 + e^{2\pi i 0.j_n} \ket 1 \right) 
  \end{equation*}
  \item 따라서 SWAP gate를 적용하여 qubit의 순서를 반대로 뒤집으면, 우리가 구하고자하는 QFT의 결과 (see Eq.~\eqref{eq:QFT-6})를 얻을 수 있다. $\Box$
\end{enumerate}

따라서 우리는 1번의 QFT circuit을 수행하는 것으로 $n$개의 data에 대한 QFT결과를 동시에 얻을 수 있다. (단, 관측하면 확률에 따라 하나의 결과만 얻게 된다.)

\subsection{Performance}
그렇다면, QFT가 가지는 gate complexity가 얼마인지 분석해보자.
알고리즘에 따라 첫 번쨰 qubit에 대해서 1개의 $H$-gate, 그리고 $(n-1)$개의 controlled rotation gate를 필요로한다. 두 번쨰 qubit에 대해서는 1개의 $H$-gate, 그리고 $(n-2)$개의 controlled rotation gate를 필요로한다. 
따라서 $n$개의 qubit에 대해 모두 필요한 gate의 개수는 다음과 같다.\footnote{qubit의 순서를 바꾸기 위해 필요한 SWAP gate에 대한 gate complexity는 $O(n)$이므로 무시할 수 있다.}
\begin{equation*}
  n + n-1 + \cdots + 1 = \frac{n(n+1)}{2} = \Theta(n^2)
\end{equation*}
\textbf{Some Remarks}
\begin{itemize}
  \item QFT는 exponential speed-up을 달성하는 것처럼 보이지만, 실제로는 값을 관측하게 되면 하나의 데이터에 대한 결과만 얻을 수 있다.
  \item 또한, input state $\ket j$를 준비하는 과정도 효과적이지 못하다.
  \item 따라서 QFT을 활용하여 알고리즘을 설계하는 것은 쉽지 않다.
\end{itemize}

\section{Phase estimation}
\subsection{Phase estimation}
QFT를 활용하는 중요한 알고리즘 중 하나가 바로 \textit{phase estimation}이다. phase estimation이 해결하고자하는 문제에 대해서 먼저 소개한다.
다음의 관계를 만족하는 unitary operator $U$에 대해, 우리가 알고 있는 것은 오직 eigenvector인 $\ket u$이고 eigenvalue는 알지 못한다고 하자.
\begin{equation*}
  U \ket u = e^{2\pi i \varphi} \ket u
\end{equation*}
Phase estimation의 목적은 eigenvalue에 대한 phase $\varphi \in [0, 1)$를 구하는 것이다.

Eigenvector를 알고있을 때, eigenvalue를 구하는 문제에는 다양한 활용이 존재한다. 예를 들어, 주어진 Hamiltonian에 대해 ground state의 energy를 추정하는 상황을 가정하자. k-local Hamiltonian $\hat H$와 ground state $\ket{\psi_0}$에 대해 다음이 성립한다.
\begin{equation*}
  \hat H \ket{\psi_0} = E_0 \ket{\psi_0}
\end{equation*}
따라서, 우리가 $E_0$를 estimate하기를 원한다는 것은 $\hat H$에 대한 eigenvector $\ket{\psi_0}$이 알려져있을 때 eigenvalue를 구하는 것을 의미하며, 이는 phase estimation으로 해결할 수 있는 문제이다.\footnote{$\hat H$를 unitary operator $\hat U = e^{-i\hat H t}$로 구성하면, $\hat U$에서 eigenvalue는 $e^{-i E_0 t}$가 된다. (with matrix exponential)}

\newpage

\begin{figure}[h]
  \centering
  \begin{minipage}{\textwidth}
    \Qcircuit @C=1.4em @R=0.9em {
        &&\lstick{\ket 0} &\gate{H} &\qw \barrier[-1.8em]{6}&\qw                      &\qw                    &\qw & \dots &       &\ctrl{6}                   &\rstick{\frac{1}{\sqrt 2}(\ket 0 + e^{2\pi i (2^{t-1} \varphi)}\ket 1)}\qw &&&&&\\
        &&                &\vdots   &                       &                       &                       &    & \dots &       &                           &           &&&\\
        &&\lstick{\ket 0} &\gate{H} &\qw                    &\qw                    &\ctrl{4}               &\qw & \dots &       &\qw                        &\rstick{\frac{1}{\sqrt 2}(\ket 0 + e^{2\pi i (2^{2} \varphi)}\ket 1)}\qw     &&&&&   \\
        &&\lstick{\ket 0} &\gate{H} &\qw                    &\ctrl{3}               &\qw                    &\qw & \dots &       &\qw                        &\rstick{\frac{1}{\sqrt 2}(\ket 0 + e^{2\pi i (2^{1} \varphi)}\ket 1)}\qw     &&&&&   \\
        &&\lstick{\ket 0} &\gate{H} &\ctrl{2}               &\qw                    &\qw                    &\qw & \dots &       &\qw                        &\rstick{\frac{1}{\sqrt 2}(\ket 0 + e^{2\pi i (2^{0}\varphi)}\ket 1)}\qw    \inputgroupv{1}{5}{0.8em}{3.6em}{t \text{ qubits} \qquad \quad} &&&&&\\
        &&                &         &                       &                       &                       &    &       &       &                           &                                                                             &&&&& \\
        &&\lstick{\ket u} &{/}\qw   &\gate{U^{2^0}}         &\gate{U^{2^1}}         &\gate{U^{2^2}}         &\qw & \dots &       &\gate{U^{2^{t-1}}}         &\rstick{\ket u}\qw &&&&&
    }
  \end{minipage}
  \caption{QPE circuit (first step)}\label{fig:QPE-circuit}
\end{figure}


Phase estimation을 설계하기 위해, 우리는 먼저 다음과 같은 oracle들을 가지고 있다고 가정한다.
\begin{itemize}
  \item eigenvector $\ket u$를 준비시킬 수 있는 oracle
  \item controlled-$U^{2^j}$ operation을 수행할 수 있는 oracle
\end{itemize}

Phase estimation algorithm은 크게 2가지 step으로 구성되는데, 첫 번째 step은 Fig.~\ref{fig:QPE-circuit}와 같이 구성되어 controlled-U operation을 차례대로 수행하는 과정이다.
두 번째 step은 첫 번째 register에 대해 \textit{inverse QFT}를 수행하는 것이다. 각 step별로 어떻게 연산이 이루어지는지 따라가보자.
\vspace{1em}

먼저 Fig.~\ref{fig:QPE-circuit}의 연산은 다음 단계를 따른다.
\begin{enumerate}
  \item input state : phase estimation이 사용하는 2개의 register들을 각각 다음과 같이 초기화한다.\footnote{첫 번째 register의 qubit 개수 $t$는 QPE의 accuracy, success probability를 결정한다.}
  \begin{equation*}
    \ket \psi = \ket {0^{\otimes t}} \ket u
  \end{equation*}
  \item apply $H$-gate : 첫 번째 register에 대해 $H$-gate를 적용하면 다음과 같은 state를 얻게된다.
  \begin{equation*}
    \ket \psi =  \frac{1}{\sqrt 2^t}(\ket 0 + \ket 1)^{\otimes t} \ket u
  \end{equation*}
  \item apply controlled-U operation : 두 번째 register를 \textit{target}으로 하고, 첫 번째 register의 각 qubit을 \textit{control}로 하여 controlled-U operation을 가하는 과정을 control qubit을 바꿔가면서 수행한다.
  
  예를 들어, $t$번째 qubit을 control qubit으로 하여 controlled-$U^{2^0}$를 가하면, 다음의 상태를 얻는다.\footnote{target qubit은 $\ket u$이지만, $U^{2^0}$s는 phase만 변화시키기 때문에, control qubit에 대해 phase term을 옮겨서 표현할 수 있다.}
  \begin{align*}
    \ket \psi &= \frac{1}{\sqrt 2^t} (\ket 0 + \ket 1)^{\otimes {t-1}} (\ket 0 \ket u + \ket 1 e^{2\pi i (2^0 \varphi)} \ket u)\\
              &=  \frac{1}{\sqrt 2^t} (\ket 0 + \ket 1)^{\otimes {t-1}} (\ket 0 + e^{2\pi i (2^0 \varphi)} \ket 1) \ket u
  \end{align*}
\end{enumerate}

일반화하여, $j$번째 qubit을 target qubit으로 하여 두 번째 레지스터에 controlled-$U^{t-j}$ gate를 가하는 과정을 $t$개의 qubit에 대해서 모두 수행하면 얻게되는 최종 state는 다음과 같다.
\begin{equation}
  \frac{1}{2^{t / 2}}\left(|0\rangle+e^{2 \pi i 2^{t-1} \varphi}|1\rangle\right)\left(|0\rangle+e^{2 \pi i 2^{t-2} \varphi}|1\rangle\right) \ldots\left(|0\rangle+e^{2 \pi i 2^0 \varphi}|1\rangle\right)=\frac{1}{2^{t / 2}} \sum_{k=0}^{2^t-1} e^{2 \pi i \varphi k}|k\rangle . \label{eq:QPE-1}
\end{equation}

이렇게 얻은 state는 \hyperref[def:QFT]{QFT}를 $ \ket{N \psi}$에 대해서 적용한 result state와 동일하다. (put $j = N\varphi$, $N=2^t$) 
따라서, $\varphi$의 값을 얻기 위하여 Eq.~\eqref{eq:QPE-1}에 inverse QFT를 수행하여 $N\varphi$를 구할 수 있다.

앞에서 우리는 $\varphi \in [0, 1)$이라고 가정하였으므로, $\varphi = 0.\varphi_1 \varphi_2 \cdots \varphi_t = \sum^t_{k=1} \varphi_k / 2^k$이라는 binary representation으로 표현할 수 있다. 
따라서 inverse QFT 결과로 얻는 $N\varphi$는 다음과 같다.
\begin{equation*}
  N\varphi = 2^t \sum^t_{k=1} \varphi_k / 2^k = \sum^t_{k=1} \varphi_k 2^{t-k} = \varphi_1\varphi_2 \cdots \varphi_t 
\end{equation*}
\newpage
즉, 최종적으로 QPE를 진행하면 $\varphi$의 binary representation에서 소수점 아래 자릿수를 값으로 갖는 $t$개의 qubit을 얻게되며, 각 qubit의 값은 0 또는 1이므로 computational basis에서 측정하게되면 $\varphi$의 근사값을 얻을 수 있다. $\Box$

\lecture{12}{28 Oct. 10:30}{}
\subsection{Performance}
앞에서 우리는 phase $\varphi$가 $t$개의 소수점 아랫수로 표현될 수 있는 값이라는 가정을 했었다.
그러나, $\varphi = 0.\varphi_1\varphi_2 \cdots \varphi_t \varphi_{t+1} \cdots$처럼 $t+1$개 이상의 소수점 아랫자리수를 가지게 되면, QPE를 사용하여 얻은 결과 $\tilde \varphi$는 실제값의 근사치가 된다. 즉, error가 발생하게 된다.

error를 분석하기 위해서, $t$개의 qubit로 나타낼 수 있는 이진수 표현중에서 가장 $\varphi$와 가까운 $\varphi$보다 크지 않은 수를 만드는 정수를 $b$라고 하자. ($b/2^t = 0.b_1 \cdots b_2$)
\begin{equation}
  b = \argmin_{0 \leq b \leq 2^t-1} \varphi - b / 2^t, \qquad (b/2^t \le \varphi) \label{eq:b}
\end{equation}
우리는 이제 QPE의 측정 결과가 $b$와 가까우며, 따라서 $\varphi$를 높은 확률로 추정할 수 있음을 보이고자한다.
Eq.~\eqref{eq:QPE-1}에 \hyperref[def:IQFT]{IQFT}를 적용한 결과는 다음과 같다.
\begin{equation*}
  \frac{1}{2^{t / 2}} \sum_{k=0}^{2^t-1} e^{2 \pi i \varphi k}|k\rangle \xrightarrow{\mathrm{IQFT}} \frac{1}{2^t} \sum_{k, l=0}^{2^t-1} e^{2 \pi i \varphi k} \underbrace{e^{-2 \pi i lk / 2^t} |l\rangle}_{\ket k} = \frac{1}{2^t} \sum_{k, l=0}^{2^t-1}\left(e^{2 \pi i\left(\varphi-l / 2^t\right)}\right)^k|l\rangle .
\end{equation*}
만약 $\varphi = l/2^t$라면, $e^{2\pi i(\varphi - l/2^t)} = 1$이 $2^t$개의 항에 대해서 더해지기 때문에, $\ket{l}$의 amplitude가 1이되어서 100\% 확률로 $\ket l = \ket{N\varphi}$를 관측한다. 그러나 $\varphi$가 $l/2^t$와 \textit{유사}하다면, 다른 항들도 관측될 확률을 가질 수 있고 이것이 바로 error를 발생시키는 원인이 된다.

\begin{definition}[Inverse Quantum Fourier transform]\label{def:IQFT}
  Inverse QFT는 fourier basis $\ket k$에 대하여, 다음 변환을 수행한다.\footnote{phase를 소거하기 위해 phase가 음수이다.}
  $$ \ket{k} \rightarrow \frac{1}{\sqrt{N}} \sum_{j=0}^{N-1} e^{-2\pi i jk/N} \ket{j}  $$
\end{definition}

\vspace{1em}
다른 항이 관측될 확률을 분석하기 위해서, $\alpha_l$을 $\ket{(b+l) \mod 2^t}$ state에 대한 amplitude라고 하자. 등비급수 공식을 사용하여 나타내면 다음과 같다. ($\delta = \varphi - b/2^t$) (See Eq.~\eqref{eq:b})
\begin{equation*}
  \alpha_l \equiv \frac{1}{2^t} \sum_{k=0}^{2^t-1}\left(e^{2 \pi i\left(\varphi-(b+l) / 2^t\right)}\right)^k=\frac{1}{2^t}\left(\frac{1-e^{2 \pi i\left(2^t \varphi-(b+l)\right)}}{1-e^{2 \pi i\left(\varphi-(b+l) / 2^t\right)}}\right)=\frac{1}{2^t}\left(\frac{1-e^{2 \pi i\left(2^t \delta-l\right)}}{1-e^{2 \pi i\left(\delta-l / 2^t\right)}}\right)
\end{equation*}
이때, $a_l$은 $b+l \mod 2$ 연산에 의해 결정되기 때문에, 서로 다른 $l$에 대해서 동일한 $\alpha_l$을 정의할 수 없도록 $l$은 $-2^{t-1} < l \le 2^{t-1}$ 범위로 제한된다.

\vspace{1em}

이제 QPE의 측정 결과가 $m$이라고 가정하자. 우리는 $m$과 $b$의 차이 $|m-b|$가 $\beta$보다 클 확률을 계산하고자한다. 이는 우리가 허용할 수 있는 error $\beta$보다 더 큰 error가 발생할 \textit{failure probability}를 의미한다.
\begin{equation}
  \Pr(|m-b| > \beta) = \sum_{\beta \le |l|} |\alpha_l|^2 = \sum_{-2^{t-1}<l \le -(\beta + 1)} |\alpha_l|^2 + \sum_{\beta+1 \le l \le 2^{t-1}}  |\alpha_1|^2 \label{eq:QPE-error-prob}
\end{equation}
이때, $|1-e^{i\theta}|$에 대한 다음의 2가지 bound를 적용하여 확률을 계산할 수 있다.
\begin{itemize}
  \item $|1-e^{i\theta}| \ge \frac{2 |\theta|}{\pi}$, for $(-\pi \le \theta \le \pi)$
  \item $|1-e^{i\theta}| \le 2$, for any $\theta \in \mathbb R$.
\end{itemize}

$l$의 범위에 대한 제한 덕분에, $-\pi \le 2\pi(\delta - l/2^t) \le \pi$ 조건 만족하여 $|\alpha_l|$의 upper bound를 구할 수 있다.
\begin{equation}
  \left|\alpha_l\right|=\left|\frac{1}{2^t}\left(\frac{1-e^{2 \pi i\left(2^t \delta-l\right)}}{1-e^{2 \pi i\left(\delta-l / 2^t\right)}}\right)\right| \leq \frac{2}{2^t | 1-e^{2 \pi i\left(\delta-l / 2^t\right)}|} \leq \frac{1}{2^{t+1}\left(\delta-l / 2^t\right)}  = \frac{1}{2} \left( \frac{1}{2^t \delta - l} \right) \label{eq:alpha-l}
\end{equation}
따라서 Eq.~\eqref{eq:QPE-error-prob}에 Eq.~\eqref{eq:alpha-l}을 대입하면, 다음의 upper bound를 얻는다.
\begin{equation}
\Pr(|m-b|>\beta) \leq \frac{1}{4}\left[\sum_{-2^{t-1}<l \leq-(\beta+1)} \frac{1}{\left(l-2^t \delta\right)^2}+\sum_{\beta+1 \leq l \leq 2^{t-1}} \frac{1}{\left(l-2^t \delta\right)^2}\right] 
\end{equation}
$0\le 2^t \delta \le 1$을 이용하여 간략히 표현하면, 다음을 얻는다.
\begin{itemize}
  \item Eq.~\eqref{eq:QPE-prob-1} : upper bound를 구하기 위해서 각 항의 값이 최대가 되도록 하는 $2^t \delta$를 대입한다.\footnote{첫 번째 항은 $l$이 음수이기 때문에 $2^t \delta$가 0이 되도록하고, 두 번째 항은 $l$이 양수이기 때문에 $2^t \delta$가 1이 되도록한다.}
  \item Eq.~\eqref{eq:QPE-prob-2} : 첫 번째 항을 보면 $l^2$에 대해서 연산하고 있기 때문에 $l$의 범위를 음수에서 양수로 바꾸어도 된다.
  \item Eq.~\eqref{eq:QPE-prob-3} : 두 항의 하나로 나타낸 뒤\footnote{using $ \frac{1}{l^2} + \frac{1}{(l-1)^2} \leq \frac{2}{l^2}. $}, 합을 적분으로 근사한다.
\end{itemize}
따라서 정리하자면, QPE의 failure probability는 $\beta$에 대해 $\frac{1}{2(\beta-1)}$로 bound 된다.
\begin{align}
  \Pr(|m-b|>\beta) & \leq \frac{1}{4}\left[\sum_{-2^{t-1}<l \leq-(\beta+1)} \frac{1}{l^2}+\sum_{\beta+1 \leq l \leq 2^{t-1}} \frac{1}{(l-1)^2}\right] \label{eq:QPE-prob-1}\\ 
  & =\frac{1}{4}\left[\sum_{\beta+1 \leq l<2^{t-1}} \frac{1}{l^2}+\sum_{\beta+1 \leq l \leq 2^{t-1}} \frac{1}{(l-1)^2}\right] \label{eq:QPE-prob-2}\\
  & \leq \frac{1}{2} \sum_{l=\beta}^{2^{t-1}-1} \frac{1}{l^2} \leq \frac{1}{2} \int_{\beta-1}^{2^{t-1}-1} \frac{1}{l^2}  dl \leq \frac{1}{2(\beta-1)} .\label{eq:QPE-prob-3}
\end{align}

만약 우리가 $\varphi$를 정확도 $2^{-n}$으로 추정하기를 원한다면,\footnote{n-bit 이진수에서 $n$번째 자릿수 이하에서 발생하는 차이값} $\beta = 2^{t-n}-1$으로 선택한다. 이때, $t$는 QPE에 사용하는 qubit의 개수이다. ($p=t-n$) 
그럼 이 $\beta$에 대해, failure probability는 $1/2(2^{t-n}-2) = 1/2(2^{p}-2)$로 bound되기 때문에, QPE의 정확도는 적어도 $1 - 1/(2(2^{p}-1))$ 이상이다. 

\vspace{1em}

따라서, $n$ bits로 표현된 $\varphi$를 success probability가 $1-\delta$ 이상이 되도록 QPE를 수행하기 위해서는, 첫 번째 레지스터의 개수 $t$를 다음과 같이 결정해야한다.
\begin{equation*}
  t = n + \left\lceil \log(2+\frac{1}{2\delta}) \right\rceil
\end{equation*}

또한, 앞에서 우리는 eigenvector의 state $\ket u$로 준비할 수 있는 oracle을 가정했다. 그러나, 실제 상황에서는 이것이 불가능할 수도 있다.
만약 우리가 다른 state $\ket \psi$를 $\ket u$ 대신에 사용했다고 하자. $\ket \psi$는 eigenbasis에 대해 다음과 같은 linear combination으로 표현된다.
\begin{equation*}
  |\psi\rangle=\sum_u c_u|u\rangle
\end{equation*}
각 eigenstate의 eigenvalue가 $e^{2\pi i \varphi_u}$라면, $\ket \psi$에 대해 phase estimation을 수행한 결과는 다음과 같이 각각의 eiganstate에 대해 phase estimation을 수행한 결과들의 superposition state가 될 것이다.
\begin{equation*}
  \ket{0^{\otimes n}} \ket \psi \xrightarrow{\text{QPE}} \sum_u c_u\left|\varphi_u\right\rangle|u\rangle,
\end{equation*}
따라서 이러한 output state를 측정하게 되면, probability $|c_u|^2$에 따라서 랜덤하게 어떤 eigenstate $\ket u$에 대한 phase $\ket {\varphi_u}$를 얻게된다. 즉, 우리가 실제로 얻고자하는 eigenstate $\ket u$에 대한 amplitude $c_u$가 충분히 크다면, QPE를 여러번 반복하여 우리가 원하는 eigenvalue를 얻을 수 있다.

\vspace{1em}
마지막으로, QPE의 resource requirement를 분석해보자. 
\begin{itemize}
  \item (space complexity) accuracy $\epsilon$을 달성할 probability가 $1-\delta$가 되도록 필요한 qubit의 개수 :
  \begin{equation*}
    t=\log (1 / \epsilon)+\left\lceil\log \left(2+\frac{1}{2 \delta}\right)\right\rceil=O\left(\log \frac{1}{\delta \epsilon}\right) .
    \end{equation*}
  \item (time complexity) 
  \begin{itemize}
    \item (step 1) QPE를 실행하기 위해서는 controlled unitary를 $2^t$번 가해야하므로, gate complexity를 qubit의 개수 $t$에 대해서 표현하면 $O(1/(\delta \epsilon))$이 된다.
    \item (step 2) 이후, $t$개의 qubit에 대해 IQFT를 가하기 때문에 IQFT의 complexity는 $\Theta(t^2) = \Theta((\log \frac{1}{\delta \epsilon})^2)$이 된다.
  \end{itemize}
  따라서 QPE의 total time complexity는 다음과 같다.
  \begin{equation*}
    O\left(\frac{1}{\delta \epsilon} \right) + \Theta\left(\left(\log \frac{1}{\delta \epsilon}\right)^2\right)
  \end{equation*}
\end{itemize}

\lecture{13}{30 Oct. 10:30}{}
\section{Applications of phase estimation}
\subsection{Ground state energy estimation}
\subsection{Order-finding algorithm}
\subsubsection{Order-finding}
\subsubsection{Uncomputation}
\subsubsection{The continued fraction expansion}
\subsubsection{Performance}

\subsection{Shor’s algorithm: factoring}


\lecture{14}{4 Nov. 10:30}{}
\section{Applications of the QFT}
\subsection{Period-finding}
\subsection{Discrete logarithm}
\subsection{Hidden subgroup problem}




\section{Quantum search algorithms}
\subsection{Grover operator}
\subsection{Grover search algorithm}
\subsection{Performance}

\lecture{15}{6 Nov. 10:30}{}
\subsection{Example: Classical circuit-SAT problem}
\subsection{Amplitude amplification}

\section{Amplitude estimation algorithm (Quantum counting)}

\section{HHL (Harrow–Hassidim–Lloyd) algorithm}

\lecture{16}{8 Nov. 17:00}{}
\section{Optimality of the quantum search algorithm}