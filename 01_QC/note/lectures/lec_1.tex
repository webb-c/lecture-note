\chapter{Quantum Algorithms}
\lecture{9}{7 Oct. 10:30}{}
\section{Introduction}
이번 챕터에서 우리는 \textit{Quantum Algorithm}에 대해 다루고자한다. Quantum algorithm은 quantum circuit이나 quantum computer에서 구현되는 알고리즘을 지칭한다.
Classical computer가 어려운 문제를 해결하기 위하여 만들어진 것처럼, quantum algorithm에 대해서 공부하고 새로운 방식을 고안하는 것은 quantum computer의 동작방식과 quantum computer의 한계를 분석하기 위한 중요한 과제이다. 
Quantum computer라는 개념이 등장하고 나서부터 지금까지 많은 종류의 quantum algorithm들이 고안되어 왔다. 이번 강의에서 다루고자하는 quantum algorithm은 다음과 같다. 
\begin{itemize}
  \item Elementary quantum algorithms
  \item Hamiltonian simulations
  \item Quantum Fourier transform
  \item Phase estimation
  \item Quantum search algorithm (Grover search algorithm)
  \item Amplitude amplification / estimation algorithms
  \item HHL algorithm
\end{itemize}

\section{Elementary quantum algorithms using quantum parallelism}
\subsection{Deutsch’s algorithm}

\lecture{10}{14 Oct. 10:30}{}
\subsection{Deutsch-Jozsa algorithm}

\section{Hamiltonian simulations}

\lecture{11}{16 Oct. 10:30}{}
\section{Quantum Fourier transform}

\section{Phase estimation}
\subsection{Phase estimation}

\lecture{12}{28 Oct. 10:30}{}
\subsection{Performance}

\lecture{13}{30 Oct. 10:30}{}
\section{Applications of phase estimation}
\subsection{Order-finding algorithm}
\subsubsection{Order-finding}
\subsubsection{Uncomputation}
\subsubsection{The continued fraction expansion}
\subsubsection{Performance}

\subsection{Shor’s algorithm: factoring}


\lecture{14}{4 Nov. 10:30}{}
\section{Applications of the QFT}
\subsection{Period-finding}
\subsection{Discrete logarithm}
\subsection{Hidden subgroup problem}




\section{Quantum search algorithms}
\subsection{Grover operator}
\subsection{Grover search algorithm}
\subsection{Performance}

\lecture{15}{6 Nov. 10:30}{}
\subsection{Example: Classical circuit-SAT problem}
\subsection{Amplitude amplification}

\section{Amplitude estimation algorithm (=Quantum counting)}

\section{HHL (Harrow–Hassidim–Lloyd) algorithm}

\lecture{16}{8 Nov. 17:00}{}
\section{Optimality of the quantum search algorithm}