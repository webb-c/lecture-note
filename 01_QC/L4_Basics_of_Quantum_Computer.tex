\documentclass[9pt]{beamer}
\usepackage{kotex}
\usepackage{amsfonts,amssymb,amsthm}
\usepackage[dvipsnames]{xcolor}
\usepackage{xcolor}
\usepackage{etoolbox}
\usepackage{braket}
\usepackage{qcircuit}

%## color
\definecolor{customBlack}{HTML}{3B4252}
\definecolor{customBlackGrey}{HTML}{434C5e}
\definecolor{cuatomGrey}{HTML}{4C566A} 
\definecolor{customWhite}{HTML}{ECEFF4} 
\definecolor{customBlue}{HTML}{6082B6}  
\definecolor{customRed}{HTML}{BF616A}
\definecolor{vividauburn}{rgb}{0.58, 0.15, 0.14}


%## Theme & custom
% \usetheme{metropolis}           % Use metropolis theme
% \metroset{block=fill}
\usetheme{moloch} % modern fork of the metropolis theme
\molochset{block=fill}
\setbeamersize{text margin left=5mm, text margin right=5mm}
\setbeamercolor{palette primary}{bg=customBlack}
\setbeamercolor{alerted text}{fg=customRed}
\setbeamercolor{itemize item}{fg=customBlue}
\setbeamercolor{enumerate item}{fg=customBlue}


%## font
\usefonttheme[onlymath]{serif}
% \setbeamerfont{normal text}{size=\small}
% \setbeamerfont{math text}{size=\tiny}


%## Theorem title, numbering
\makeatletter
\setbeamertemplate{theorem begin}
{%
\begin{\inserttheoremblockenv}
{%
\inserttheoremheadfont
\inserttheoremname
\ifx\inserttheoremaddition\@empty\else\ of\ \inserttheoremaddition\fi%
\inserttheorempunctuation
}%
}
\setbeamertemplate{theorem end}{\end{\inserttheoremblockenv}}
\makeatother
\setbeamertemplate{theorems}[numbered]  


%## Custom block
\setbeamercolor{block title}{bg=customBlue, fg=white}
\setbeamercolor{block body}{bg=customWhite, fg=customBlack}
\setbeamercolor{block title alerted}{%
    use={block title, alerted text},
    bg=customRed,
    fg=white
}
\setbeamercolor{block body alerted}{%
    use={block title, alerted text},
    bg=customWhite,
    fg=customBlack
}
\AtBeginEnvironment{definition}{%
    \setbeamercolor{block title}{fg=white,bg=customBlackGrey}
    \setbeamercolor{block body}{fg=customBlack, bg=customWhite}
}
\AtBeginEnvironment{theorem}{%
    \setbeamercolor{block title}{fg=white,bg=customBlackGrey}
    \setbeamercolor{block body}{fg=customBlack, bg=customWhite}
}
\AtBeginEnvironment{corollary}{%
    \setbeamercolor{block title}{fg=white,bg=customBlackGrey}
    \setbeamercolor{block body}{fg=customBlack, bg=customWhite}
}
\AtBeginEnvironment{lemma}{%
    \setbeamercolor{block title}{fg=white,bg=customBlackGrey}
    \setbeamercolor{block body}{fg=customBlack, bg=customWhite}
}


%! Useful command
\renewcommand{\Pr}{\text{Pr}}
% $\ast$ \underline{Proof}:
%\checkmark \underline{meaning}:

\title{4. Basics of Quantum Computer}
\date{\today}
\author{Vaughan Sohn}
% \institute{Centre for Modern Beamer Themes}


\begin{document}
    %#################################### 
    \maketitle
    
    %#################################### 
    \begin{frame}
        \frametitle{Contents}
        \tableofcontents
    \end{frame}


    %#################################### 
    \begin{section}{Quantum Circuit model}
        \begin{frame}
            \frametitle{Component of quantum circuit model}
        
        \end{frame}

        \begin{frame}
            \frametitle{Qubit}
        
        \end{frame}
        
    \end{section}

    %#################################### 
    \begin{section}{Quantum Gates}
        \begin{frame}
            \frametitle{Single-qubit gate}
        
        
        \end{frame}
        \begin{frame}
            \frametitle{Single-qubit gate decomposition}
                \begin{theorem}[ZY decomposition]\label{thr:ZY-de}
                    Suppose $U$ is a unitary operation on a single qubit. Then there exist real numbers $\alpha, \beta, \gamma$ and $\delta$ such that,
                    $$U=e^{i \alpha} R_z(\beta) R_y(\gamma) R_z(\delta)$$
                \end{theorem}
                
                \begin{theorem}\label{thr:Pauli-de}
                    Suppose $U$ is a unitary gate on a single qubit. Then there exist unitary operators $A, B, C$ on a single qubit such that $ABC = I$ and
                    $$ U = e^{i\alpha} AXBXC $$
                    where $\alpha$ is some overall phase factor and $X$ is a Pauli-X operator.
                \end{theorem}

        \end{frame}

        \begin{frame}
            \frametitle{Single-qubit gate decomposition}
                $\ast$ \underline{Proof}:
                
        \end{frame}

        \begin{frame}
            \frametitle{Controlled gate}

            \begin{table}[h]
            \[
            \begin{array}{c}
            \Qcircuit @C=1.5em @R=.8em {
                & \ctrl{2} & \targ & \gate{U} & \qw \\
                & \qw & \ctrl{-1} & \qw & \qw \\
                & \targ & \ctrl{-1} & \ctrl{-2} & \qw \\
                & \qw & \ctrl{-1} & \qw & \qw
            }
            \end{array}
            \]
            \caption{Quantum Circuit} \label{fig:my_label} 
            \end{table}
        \end{frame}

        \begin{frame}
            \frametitle{Controlled gate decomposition}
        
            
        
        \end{frame}
        \begin{frame}
            \frametitle{Controlled gate on multiple-qubit}
        
            
        
        \end{frame}
        \begin{frame}
            \frametitle{Controlled gate on multiple-qubit decomposition}
        
            
        
        \end{frame}

        \begin{frame}
            \frametitle{Summary}
            \begin{block}{Summary}
                
            \end{block}
            \vspace{0.2cm}
            \textit{Some remarks}
            \begin{itemize}
                \item 
            \end{itemize}
            
        
        \end{frame}

    \end{section}

    %#################################### 
    \begin{section}{Universial Quantum gate set: \{CNOT, single qubit gates\}}
        \begin{frame}
            \frametitle{Decomposition from n-qubit unitary gate to two-level unitary gates}
        
            
        
        \end{frame}
        \begin{frame}
            \frametitle{Two-level unitary gate is controlled-U gate}
                \begin{theorem}\label{thr:two-level de}
                    Unitary operator $U$ which acts on a $d$-dimensional Hilbert space may be decomposed into a product of two-level unitary matrices;
                \end{theorem}
                \vspace{0.2cm}
                $\ast$ \underline{Proof}: from $3\times 3$ example,
        
        \end{frame}
        \begin{frame}
            \frametitle{Two-level unitary gate is controlled-U gate}
                $\ast$ \underline{Proof}: (contd.)
        
        \end{frame}

        \begin{frame}
            \frametitle{Decomposition from n-qubit controlled-U gate to \{CNOT gates, single-qubit\}}
            \framesubtitle{$\Rightarrow$ Single qubit and CNOT gates are universal!}
            \begin{theorem}\label{thr:CNOT-single de}
                $n$-qubit controlled-$U$ gate can be decomposed into a single qubit gate and CNOT gates;
            \end{theorem}
            \vspace{0.2cm}
            $\ast$ \underline{Proof}:
        \end{frame}

        \begin{frame}
            \frametitle{Decomposition from n-qubit controlled-U gate to \{CNOT gates, single-qubit\}}
            \framesubtitle{$\Rightarrow$ Single qubit and CNOT gates are universal!}
                $\ast$ \underline{Proof}: (contd.)
        
        \end{frame}


        \begin{frame}
            \frametitle{Decomposition from n-qubit controlled-U gate to \{CNOT gates, single-qubit\}}
            \framesubtitle{$\Rightarrow$ Single qubit and CNOT gates are universal!}
            \begin{corollary}\label{col:universial-ver1}
                single qubit and CNOT gates together can be used to implement an \alert{arbitrary $n$-qubit unitary operation}.
            \end{corollary}
            $\ast$ \underline{Proof}: Combine theorem \ref{thr:two-level de} and \ref{thr:CNOT-single de}, we can easily proof this corollary.$\Box$
        \end{frame}

        \begin{frame}
            \frametitle{Circuit complexity}
        
            
        
        \end{frame}

        \begin{frame}
            \frametitle{Summary}
            \begin{block}{Summary}
                
            \end{block}
            \vspace{0.2cm}
            \textit{Some remarks}
            \begin{itemize}
                \item 
            \end{itemize}
        
        \end{frame}
    \end{section}
    
    %#################################### 
    % rotation이랑 다르게 discrete gate들임
    \begin{section}{Universial Quantum Discrete gate set: \{CNOT, H, S, T\}}

        \begin{frame}
            \frametitle{Approximating quantum circuits via discrete gate set}
                \begin{definition}[approximation error]\label{def:error}
                    We define the \textbf{error} when $V$ is implemented instead of $U$ by
                    $$E(U,V) \triangleq \max_{\ket{\psi}} \|(U-V) \ket{\psi}\|$$
                    where the maximum is over all normalized quantum states $\ket{\psi}$ in the state space.
                \end{definition}
                \vspace{0.2cm}
                \checkmark \underline{meaning}:
            
        
        \end{frame}

        \begin{frame}
            \frametitle{Approximating quantum circuits via discrete gate set}
                \begin{definition}[variational distance]\label{def:vari-dist}
                    variational distance as 
                    $$VD(P_U(m), P_V(m)) = \frac{1}{2} |P_U(m) - P_V(m)|$$
                    and total variational distance 
                    $$TVD(P_U, P_V) = \frac{1}{2} \sum_m |P_U(m) - P_V(m) |$$
                \end{definition}
                \vspace{0.2cm}
                \checkmark \underline{meaning}:
            
        
        \end{frame}

        \begin{frame}
            \frametitle{Approximating quantum circuits via discrete gate set}
                \begin{theorem}[quantum gate error bound]\label{thr:gate-error}
                    $$ \left|P_U(m)-P_V(m)\right| \leq 2 E(U, V) $$
                \end{theorem}
                \vspace{0.2cm}
                $\ast$ \underline{Proof}:

        
        \end{frame}

        \begin{frame}
            \frametitle{Approximating quantum circuits via discrete gate set}
                \begin{theorem}[quantum circuit error bound]\label{thr:circuit-error}
                    $$ E\left(U_m U_{m-1} \ldots U_1, V_m V_{m-1} \ldots V_1\right) \leq \sum_{j=1}^m E\left(U_j, V_j\right)$$
                \end{theorem}
                \vspace{0.2cm}
                $\ast$ \underline{Proof}:

        
        \end{frame}

        \begin{frame}
            \frametitle{Generate two type of rotational gate $R_{{\hat n}}(\hat \theta), R_{\hat m}(\hat \theta)$}
        
            
        
        \end{frame}

        \begin{frame}
            \frametitle{Approximation error in arbitrary rotation gate is bounded by $\epsilon$}
            \begin{theorem}
                We can implement $V$ via $\{H, T, S\}$ that satisfy following bound
                $$ E(U, V) \leq \epsilon, $$
                where $\epsilon$ is target error rate.
                \vspace{0.2cm}
                $\ast$ \underline{Proof}: (hint) using kronecker theorem
            \end{theorem}
            
        
        \end{frame}

        \begin{frame}
            \frametitle{Approximating n-qubit unitary gate via ${R_{\hat n}}(\hat \theta), R_{\hat m}(\hat \theta)$}
            \framesubtitle{$\Rightarrow$ H, S, T and CNOT gates are universal!}
            
        
        \end{frame}

        \begin{frame}
            \frametitle{Circuit complexity: for \# of single qubit gates}
            \begin{theorem}[Solovay Kitaev theorem]
                $$n_1 = O\left(\log^c \left( \frac{1}{\epsilon_1} \right)\right) = O\left(\log^c \left( \frac{m}{\epsilon} \right)\right) $$
                전체에 대해서는 
                $$ m \times O\left(\log^c \left( \frac{m}{\epsilon} \right)\right) = O(m \log^c m)$$
            \end{theorem}
        \end{frame}


        \begin{frame}
            \frametitle{Circuit complexity: for \# of qubits}
            \begin{theorem}
                For implement arbitrary $n$-qubit unitary gate $U$ needs $\Omega(2^n)$ number of gates.
            \end{theorem}
            \vspace{0.2cm}
            $\ast$ \underline{Proof}: $U$가 만들어낼 수 있는 $\ket \psi$의 경우의 수를 이용한다.
            \\ \textbf{method 1}
        \end{frame}

        \begin{frame}
            \frametitle{Circuit complexity: for \# of qubits}
            \textbf{method 2}
        \end{frame}

        \begin{frame}
            \frametitle{Summary}
            \begin{block}{Summary}
                
            \end{block}
            \vspace{0.2cm}
            \textit{Some remarks}
            \begin{itemize}
                \item 
            \end{itemize}
        
        \end{frame}
    \end{section}
    
    %#################################### 
    \begin{section}{Measurement}
        \begin{frame}
            \frametitle{Principle of deferred measurement}
            Computational basis: $\{\ket 0 \bra 0 , \ket 1 \bra 1\}$
            \begin{block}{Principle of deferred measurement}
                Measurements can always be moved from an intermediate stage of a quantum circuit to the end of the circuit; if the measurement results are used at any stage of the circuit then the classically controlled operations can be replaced by conditional quantum operations.
            \end{block}
            \vspace{0.2cm}
            \checkmark \underline{meaning}:
        \end{frame}

        \begin{frame}
            \frametitle{Principle of implicit measurement}
            \begin{block}{Principle of implicit measurement}
                Without loss of generality, any  unterminated quantum wires (qubits which are not measured) at the end of a quantum circuit may be assumed to be measured.
            \end{block}
            \vspace{0.2cm}
            \checkmark \underline{meaning}:
        
        \end{frame}
    \end{section}
    

    
    \begin{frame}{References}
        
        \begin{itemize}
            \item M. A. Nielson and I. L. Chuang, Quantum Computation and Quantum Information
            \item Lecture notes for QU511: Quantum Computing (Fall 2024)
        \end{itemize}
        \vspace{6cm}
    \end{frame}

\end{document}