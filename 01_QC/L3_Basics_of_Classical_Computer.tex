\documentclass[9pt]{beamer}
\usepackage{kotex}
\usepackage{amsfonts,amssymb,amsthm}
\usepackage[dvipsnames]{xcolor}
\usepackage{xcolor}
\usepackage{etoolbox}
\usepackage{braket}
%## color
\definecolor{customBlack}{HTML}{3B4252}
\definecolor{customBlackGrey}{HTML}{434C5e}
\definecolor{cuatomGrey}{HTML}{4C566A} 
\definecolor{customWhite}{HTML}{ECEFF4} 
\definecolor{customBlue}{HTML}{6082B6}  
\definecolor{customRed}{HTML}{BF616A}
\definecolor{vividauburn}{rgb}{0.58, 0.15, 0.14}


%## Theme & custom
% \usetheme{metropolis}           % Use metropolis theme
% \metroset{block=fill}
\usetheme{moloch} % modern fork of the metropolis theme
\molochset{block=fill}
\setbeamersize{text margin left=5mm, text margin right=5mm}
\setbeamercolor{palette primary}{bg=customBlack}
\setbeamercolor{alerted text}{fg=customRed}
\setbeamercolor{itemize item}{fg=customBlue}


%## font
\usefonttheme[onlymath]{serif}
% \setbeamerfont{normal text}{size=\small}
% \setbeamerfont{math text}{size=\tiny}


%## Theorem title, numbering
\makeatletter
\setbeamertemplate{theorem begin}
{%
\begin{\inserttheoremblockenv}
{%
\inserttheoremheadfont
\inserttheoremname
\ifx\inserttheoremaddition\@empty\else\ of\ \inserttheoremaddition\fi%
\inserttheorempunctuation
}%
}
\setbeamertemplate{theorem end}{\end{\inserttheoremblockenv}}
\makeatother
\setbeamertemplate{theorems}[numbered]  


%## Custom block
\setbeamercolor{block title}{bg=customBlue, fg=white}
\setbeamercolor{block body}{bg=customWhite, fg=customBlack}
\setbeamercolor{block title alerted}{%
    use={block title, alerted text},
    bg=customRed,
    fg=white
}
\setbeamercolor{block body alerted}{%
    use={block title, alerted text},
    bg=customWhite,
    fg=customBlack
}
\AtBeginEnvironment{definition}{%
    \setbeamercolor{block title}{fg=white,bg=customBlackGrey}
    \setbeamercolor{block body}{fg=customBlack, bg=customWhite}
}
\AtBeginEnvironment{theorem}{%
    \setbeamercolor{block title}{fg=white,bg=customBlackGrey}
    \setbeamercolor{block body}{fg=customBlack, bg=customWhite}
}
\AtBeginEnvironment{corollary}{%
    \setbeamercolor{block title}{fg=white,bg=customBlackGrey}
    \setbeamercolor{block body}{fg=customBlack, bg=customWhite}
}
\AtBeginEnvironment{lemma}{%
    \setbeamercolor{block title}{fg=white,bg=customBlackGrey}
    \setbeamercolor{block body}{fg=customBlack, bg=customWhite}
}


%! Useful command
\renewcommand{\Pr}{\text{Pr}}
% $\ast$ \underline{Proof}:
%\checkmark \underline{meaning}:

\title{2. Postulates on quantum computing}
\date{\today}
\author{Vaughan Sohn}
% \institute{Centre for Modern Beamer Themes}


\begin{document}
    %#################################### 
    \maketitle
    
    %#################################### 
    \begin{frame}
        \frametitle{Contents}
        \tableofcontents
    \end{frame}


    %#################################### 
    \begin{section}{Turing machine}
        \begin{frame}{Definition of Turing machine}
            \textbf{Components of a Turing machine}
            \begin{itemize}
                \item 
            \end{itemize}
        \end{frame}

        \begin{frame}{Definition of Turing machine}
            \textbf{Operation of a Turing Machine}
            \begin{itemize}
                \item 
            \end{itemize}
        \end{frame}

        \begin{frame}{Universial Turing machine}
            \begin{itemize}
                \item 
            \end{itemize}
        \end{frame}

        \begin{frame}{Church-Turing thesis}
            \begin{definition}
                A partial function $f: A^* \rightarrow A$ is computable if there exists a Turing machine $M$ such that $\delta_M = f$. In this case, we say that $f$ is computed by $M$.
            \end{definition}
            \begin{block}{Church-Turing thesis}
                The class of functions computable by a Turing machine corresponds exactly to the class of functions which we would naturally regard as being computable by an algorithm.
            \end{block}
            \begin{itemize}
                \item 
            \end{itemize}
        \end{frame}
        
        \begin{frame}{Halting problem}
            \begin{alertblock}{Halting problem}
                Dose turing machine $M$ halt for given input $x$?
                \\ $\rightarrow$ \textit{We can't compute halting problem by any turing machine!}
            \end{alertblock}
            \vspace{0.2cm}
            $\ast$ \underline{Proof}:
            (귀류법) Halting 문제를 풀 수 있는 TM $\text{HALT}$가 존재한다고 가정하자.
            \vspace{5cm}

        \end{frame}

    \end{section}


    \begin{section}{Circuit model}
        \begin{frame}{Definition of Circuit model}
            \begin{itemize}
                \item 
            \end{itemize}
        \end{frame}

        \begin{frame}{Universiality of Circuit model}
            \begin{theorem}
                Circuit model can solve every type of boolean function.
                $$f : \{0, 1\}^n \rightarrow \{0, 1\}^m$$
            \end{theorem}
        \end{frame}
    \end{section}

    \begin{section}{Two computation model}
        \begin{frame}{Association between two computation models}
            %! TODO
            \begin{definition}[uniform circuit family]
                
            \end{definition}
            \begin{itemize}
                \item 
            \end{itemize}
            \begin{alertblock}{Can circuit model solve halting problem?}
                
            \end{alertblock}
        \end{frame}
    \end{section}


    %#################################### 
    \begin{frame}{References}
        
        \begin{itemize}
            \item M. A. Nielson and I. L. Chuang, Quantum Computation and Quantum Information
            \item Lecture notes for QU511: Quantum Computing (Fall 2024)
        \end{itemize}
        \vspace{6cm}
    \end{frame}

\end{document}