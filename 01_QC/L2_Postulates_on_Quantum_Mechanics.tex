\documentclass[9pt]{beamer}
\usepackage{kotex}
\usepackage{amsfonts,amssymb,amsthm}
\usepackage[dvipsnames]{xcolor}
\usepackage{xcolor}
\usepackage{etoolbox}
\usepackage{braket}
%## color
\definecolor{customBlack}{HTML}{3B4252}
\definecolor{customBlackGrey}{HTML}{434C5e}
\definecolor{cuatomGrey}{HTML}{4C566A} 
\definecolor{customWhite}{HTML}{ECEFF4} 
\definecolor{customBlue}{HTML}{6082B6}  
\definecolor{customRed}{HTML}{BF616A}
\definecolor{vividauburn}{rgb}{0.58, 0.15, 0.14}


%## Theme & custom
% \usetheme{metropolis}           % Use metropolis theme
% \metroset{block=fill}
\usetheme{moloch} % modern fork of the metropolis theme
\molochset{block=fill}
\setbeamersize{text margin left=5mm, text margin right=5mm}
\setbeamercolor{palette primary}{bg=customBlack}
\setbeamercolor{alerted text}{fg=customRed}
\setbeamercolor{itemize item}{fg=customBlue}


%## font
\usefonttheme[onlymath]{serif}
% \setbeamerfont{normal text}{size=\small}
% \setbeamerfont{math text}{size=\tiny}


%## Theorem title, numbering
\makeatletter
\setbeamertemplate{theorem begin}
{%
\begin{\inserttheoremblockenv}
{%
\inserttheoremheadfont
\inserttheoremname
\ifx\inserttheoremaddition\@empty\else\ of\ \inserttheoremaddition\fi%
\inserttheorempunctuation
}%
}
\setbeamertemplate{theorem end}{\end{\inserttheoremblockenv}}
\makeatother
\setbeamertemplate{theorems}[numbered]  


%## Custom block
\setbeamercolor{block title}{bg=customBlue, fg=white}
\setbeamercolor{block body}{bg=customWhite, fg=customBlack}
\setbeamercolor{block title alerted}{%
    use={block title, alerted text},
    bg=customRed,
    fg=white
}
\setbeamercolor{block body alerted}{%
    use={block title, alerted text},
    bg=customWhite,
    fg=customBlack
}
\AtBeginEnvironment{definition}{%
    \setbeamercolor{block title}{fg=white,bg=customBlackGrey}
    \setbeamercolor{block body}{fg=customBlack, bg=customWhite}
}
\AtBeginEnvironment{theorem}{%
    \setbeamercolor{block title}{fg=white,bg=customBlackGrey}
    \setbeamercolor{block body}{fg=customBlack, bg=customWhite}
}
\AtBeginEnvironment{corollary}{%
    \setbeamercolor{block title}{fg=white,bg=customBlackGrey}
    \setbeamercolor{block body}{fg=customBlack, bg=customWhite}
}
\AtBeginEnvironment{lemma}{%
    \setbeamercolor{block title}{fg=white,bg=customBlackGrey}
    \setbeamercolor{block body}{fg=customBlack, bg=customWhite}
}


%! Useful command
\renewcommand{\Pr}{\text{Pr}}
% $\ast$ \underline{Proof}:
%\checkmark \underline{meaning}:

\title{2. Postulates on quantum mechanics}
\date{\today}
\author{Vaughan Sohn}
% \institute{Centre for Modern Beamer Themes}


\begin{document}
    %#################################### 
    \maketitle
    
    %#################################### 
    \begin{frame}
        \frametitle{Contents}
        \tableofcontents
    \end{frame}

    %#################################### 
    \begin{frame}{Introduction}
        \begin{itemize}
            \item 양자역학은 미시세계의 물리학을 다루는 학문이다. 
            \item 양자역학을 다루기 위해서는 물리학적 개념을 우리에게 익숙한 언어인 "\alert{수학}"으로 표기할 필요가 있다.
            \item 양자역학의 공준(postulate)은 다음 4가지 개념을 수학적으로 어떻게 정의할 것인지를 언급한다.
            \begin{itemize}
                \item state
                \item dynamics
                \item measurement
                \item composite system
            \end{itemize}
        \end{itemize}
    \end{frame}
    
    %#################################### 
    \begin{section}{State}
        \begin{frame}{State: state vector}
            \begin{block}{Postulate 1}
                Associated to any closed physical system is a complex vector space with inner product (i.e., \textit{Hilbert space}) known as the state space of the system. The system is completely described by its state vector, which is a unit vector in the system’s state space.
                $$ \ket \psi \in \mathcal H$$
            \end{block}
            \begin{itemize}
                \item 계의 상태를 기술하기 위하여 state vector를 사용한다. (unit vector)
                \item 가능한 모든 상태들의 집합; 상태공간은 Hilbert space이다.
                \item 상태 벡터를 사용하면 계를 \textbf{완전히} 기술할 수 있다.
                \item (qubit) $\mathcal H_2$의 basis가 $\{\ket 0, \ket 1\}$일 때, state vector는 다음과 같이 표현할 수 있다.
                $$\ket \psi = a \ket 0 + b \ket 1$$
                where $|a|^2 + |b|^2 = 1$.
            \end{itemize}
        \end{frame}
    \end{section}

    %#################################### 
    \begin{section}{Dynamics}
        \begin{frame}{Dynamics: unitary operator}
            \begin{block}{Postulate 2}
                The evolution of a closed quantum system is described by a \textbf{unitary transformation}. That is, the state $\ket \psi$ of the system at time $t$ is related to the state $\ket{\psi '}$ of the system at time $t_0$ by a unitary operator $U$ which depends \textit{only} on the times $t_0$ and $t$,
                $$ \ket {\psi'}  = U \ket \psi.$$
            \end{block}
            \begin{block}{Postulate 2'}
                The time evolution of the state of a closed quantum system is described by the \textit{Schrödinger equation}, $(\ast)$
                $$ i \hbar \frac{d \ket {\psi(t)}}{ dt} = \mathbf{H} \ket{\psi (t)}.$$
                where $\mathbf H$ is Hamiltonian operator.
            \end{block}
            \begin{itemize}
                \item Unitary operator는 특정한 discrete time $t_0, t$에 대한 dynamics를 나타내지만, Schrödinger equation은 continuous time에 대한 변화를 표현한다.
                \item Hamiltonian은 계의 "에너지"에 대한 operator이다. 즉, quantum state의 에너지를 Hamiltonian operator로 변환하여 state의 evolution을 이끌어낼 수 있다.
            \end{itemize}
        \end{frame}
            
        \begin{frame}{Dynamics in qubit system}
            \begin{itemize}
                \item Pauli operator:
                \\Pauli operator를 spectral decomposition해서 표현하면 다음과 같다.
                \begin{itemize}
                    \item (bit flip) $X = \ket + \bra + - \ket - \bra - (=\ket 0 \bra 1 + \ket 1 \bra 0 )$
                    \item (phase flip) $Z = \ket 0 \bra 0 - \ket 1 \bra 1$
                    \item $Y = \ket i \bra i - \ket {-i} \bra {-i}$
                \end{itemize}
                \vspace{0.5cm}
                 \item Hadamard gate:
                \\ basis state $\ket 0, \ket 1$와 superposition state $\ket +, \ket -$를 서로 변환한다.
                $$ H = \frac{1}{\sqrt 2} \begin{pmatrix} 1 & 1 \\ 1 & -1\end{pmatrix}$$
            \end{itemize}
        \end{frame}
    \end{section}

    %#################################### 
    \begin{section}{Measurement}
        \begin{frame}{Measurement: linear operator}
            \begin{block}{Postulate 3}
                Quantum measurement is described by set of measurement operators $\{M_m\}$ where $\sum_m M_m^\dagger M_m = I$ (completeness relation). 
                \\ The \textit{probability} of getting outcome $m$ is given by
                $$p(m) = \braket{\psi | M_m^\dagger M_m | \psi}, $$
                and the \textit{post-measurement state} is given by
                $$\frac{M_m\ket\psi}{ \sqrt{\braket{\psi | M_m^\dagger M_m | \psi}}} = \frac{M_m\ket\psi}{ \sqrt{p(m)}}.$$
            \end{block}
            \begin{itemize}
                \item 계에서 어떤 일이 일어났는지를 확인하기 위해서는 "관측"을 해야하며, 관측을 하게되는 순간 외부와 상호작용을 하게되므로 그 계는 더 이상 닫힌계가 아니다.
                \item measurement operator가 \textit{completeness equation}을 만족시키기 때문에 각 outcome들의 확률의 합은 1이다. $(\ast)$
                \item Example: $\ket \psi = a \ket 0 + b \ket 1$ 일 때, $p(0), p(1)$은?
                \vspace{1cm}
            \end{itemize}
        \end{frame}

        \begin{frame}{Projective measurements}
            \begin{definition}[projective measurement]
                Projective measurement is described by set of projector $\{P_m\}$. We first define \textit{observable} $M$, a Hermitian operator on the state space of the system being observed.
                $M$ has a spectral decomposition
                $$ M = \sum_m m P_m = \sum_m m \ket m \bra m, $$
                where $P_m$ is the projector onto the eigenspace of $M$ with eigenvalue $m$.
            \end{definition}
            \begin{itemize}
                \item 일반적인 관측에서는 operator $M_m$이 completeness relation을 만족하기만 하면되지만, projective measurement에서는 $M_m$이 Hermitian이고 $M_m M_{m'} = \delta_{m,m'} M_m$라는 조건을 만족해야하는 특수한 경우를 다룬다.
                \item The \textit{probability} of getting outcome $m$ is given by $(ast)$
                $$p(m) = \braket{\psi | P_m | \psi}, $$
                \item and the \textit{post-measurement state} is given by
                $$\frac{P_m\ket\psi}{ \sqrt{\braket{\psi | P_m | \psi}}} = \frac{P_m\ket\psi}{ \sqrt{p(m)}}.$$
            \end{itemize}
        \end{frame}

        \begin{frame}{POVM measurements}
            \begin{definition}[POVM measurement]
                Suppose a measurement described by measurement operators $M_m$ is performed upon a quantum system in the state $\ket \psi$. Then the probability of outcome $m$ is given by $p(m) = \braket{\psi | M_m^\dagger M_m | \psi}$. Suppose we define
                $$E_m \triangleq M_m^\dagger M_m$$ 
                as positive operator.
                then probability can be simply described by $p(m) = \braket{\psi | E_m | \psi}.$
                The operators $E_m$ are known as the \textbf{POVM elements} associated with the measurement. The complete set $\{E_m\}$ is known as a \textbf{POVM}.
            \end{definition}
            \begin{itemize}
                \item POVM formalism을 사용하면, 계를 측정할 때의 확률을 분석하기 적합하다.
                \item POVM은 completeness relation을 만족한다.
                $$\sum_m E_m = I$$
            \end{itemize}
        \end{frame}

        \begin{frame}{Examples}
            \begin{itemize}
                \item Example 1: 다음의 measurement operator $\{M_i\}$를 사용하여 $\ket 0$을 측정할 때, 각 outcome의 확률을 계산하라.
                $$M_1 = \frac{\ket 0 \bra 0}{\sqrt 2}, \quad M_2 = \frac{\ket 1 \bra 1}{\sqrt 2}, \quad M_3 = \frac{\ket + \bra +}{\sqrt 2}, \quad M_4 = \frac{\ket - \bra -}{\sqrt 2}$$
                \vspace{1cm}
                
                \item Example 2: $\{\ket 0 \bra 0, \ket 1 \bra 1\}$을 사용하여 $\ket +$을 측정할 때, 확률을 계산하라.
                
                \vspace{1.3cm}

                \item Example 3: $\{\ket + \bra +, \ket - \bra -\}$을 사용하여 $\ket +$을 측정할 때, 확률을 계산하라.
            
            \end{itemize}
        \end{frame}

        \begin{frame}{Distinguishing qunatum states: orthogonal states}
            \begin{theorem}
                If the states $\ket{\psi_i}$ are orthonormal then we can prove that there always exists a quantum measurement capable of distinguishing the states.
            \end{theorem}
            $\ast$ \underline{Proof}:
            \begin{itemize}
                \item 서로 수직인 두 상태를 가정하자. $\braket{\psi | \psi^{\perp}} = 0$
                \item 두 상태를 구분하기 위해서 다음과 같이 POVM을 가정하자.
                $$E_1 = \ket{\psi} \bra{\psi}, \quad E_2 = \ket{\psi^{\perp}} \bra{\psi^{\perp}}, \quad E_3 = I-(E_1 + E_2)$$
                \item 그럼 각 상태에 대해, 가능한 모든 outcome $m$의 확률은 다음과 같다. 
                \\ $\Rightarrow$
                \vspace{3cm}
            \end{itemize}
            따라서 서로 수직인 두 상태는 측정을 통해 두 상태를 완벽하게 구분할 수 있다.$\Box$
        \end{frame}

        \begin{frame}{Distinguishing qunatum states: non-orthogonal states}
            \begin{theorem}
                If the states $\ket{\psi_i}$ are not orthonormal then we can prove that there is no quantum measurement capable of distinguishing the states.
            \end{theorem}
            $\ast$ \underline{Proof}:
            \begin{itemize}
                \item 서로 수직이 아닌 두 상태를 가정하자. $\braket{\psi | \psi'} \ne 0$
                \item 하나의 상태를 다음과 같이 linear combination으로 나타낼 수 있다. ($a \ne 0, b \ne 1$)
                \begin{equation} \ket {\psi'} = a \ket \psi + b \ket{\psi^\perp} \end{equation}
                \item (귀류법) 두 상태를 완벽하게 구분할 수 있는 POVM이 존재한다고 가정하자.
                $$E_1 = M_1^\dagger M_1, \quad E_2 = M_2^\dagger M_2, \quad I = E_1 + E_2$$
                \item 두 상태를 완벽하게 구분할 수 있으려면 각 상태가 서로 다른 outcome에 대해 deterministic하게 동작해야한다. 
                \begin{equation}\braket{\psi|E_1|\psi} = 1,\quad \braket{\psi'|E_2|\psi'} = 1 \end{equation}
                \item Eq 1, 2를 이용하여 식을 전개해나가면 모순을 찾을 수 있다.
            \end{itemize}
        \end{frame}

        \begin{frame}{Distinguishing qunatum states: non-orthogonal states}
            $\ast$ \underline{Proof} (contd.):
            
            $$\braket{\psi|I|\psi} = \qquad \qquad \qquad \qquad \qquad \qquad \qquad \qquad\qquad \qquad \qquad $$
            \vspace{0.5cm}
            $$\sqrt{E_2} \ket{\psi'} = \qquad \qquad \qquad \qquad \qquad \qquad \qquad \qquad\qquad \qquad \qquad $$
            \vspace{0.5cm}
            $$\braket{\psi^\perp|I|\psi^\perp} = \qquad \qquad \qquad \qquad \qquad \qquad \qquad \qquad\qquad \qquad  \qquad $$
            \vspace{0.5cm}
            $$\braket{\psi'|E_2|\psi'} =\qquad \qquad \qquad \qquad \qquad \qquad \qquad \qquad\qquad \qquad \qquad  $$
            \vspace{0.3cm}
            $$1 \le |b|^2 < 1$$
            
            \vspace{0.1cm}

            따라서 서로 수직이 아닌 두 상태를 측정을 통해 완벽하게 구분할 수는 없다.$\Box$ (단, partially distinguishing은 가능하다.)
        \end{frame}

        \begin{frame}{Phase}
            Global phase:
            \begin{itemize}
                \item $\ket{\psi}, e^{i\theta}\ket{\psi}$라는 두 상태에 대해서는 어떤 measurement $M_i$에 대해서도 확률이 동일하다.
                \\ $\ast$ Proof:
                \begin{itemize}
                    \item $p(i)_{\ket{\psi}} = \braket{\psi|M_i^\dagger M_i|\psi}$
                    \item $p(i)_{\ket{e^{i\theta}\psi}} = \braket{\psi|e^{i\theta}M_i^\dagger M_ie^{-i\theta}|\psi} = \braket{\psi|M_i^\dagger M_i|\psi}$
                \end{itemize}
                \item 즉, global phase가 곱해지더라도 두 상태는 physically 동일하다.
            \end{itemize}
            
            \vspace{1cm}
            Relative phase:
            \begin{itemize}
                \item 반면 linear combination의 일부항에만 phase가 곱해지는 relative phase는 구분된다.
                \item example:
                \begin{itemize}
                    \item $\ket{\psi_1} = \frac{1}{\sqrt 2} (\ket0 + \ket 1) = \ket{+}$
                    \item $\ket{\psi_2} = \frac{1}{\sqrt 2} (\ket 0 - \ket 1) = \frac{1}{\sqrt 2} (\ket 0 + (e^{i\pi}) \ket 1) =\ket{-}$
                \end{itemize}
            \end{itemize}

        \end{frame}
    \end{section}

    %#################################### 
    \begin{section}{Composite system}
        \begin{frame}{Composite system: tensor product}
            \begin{block}{Postulate 4}
                The state space of a composite physical system is the tensor product of the state spaces of the component physical systems. Moreover, if we have systems numbered $1$ through $n$, and system number $i$ is prepared in the state $\ket{\psi_i}$, then the joint state of the total system is $$\ket{\psi_1} \otimes \ket{\psi_2} \otimes \cdots \otimes \ket{\psi_n}.$$
            \end{block}
            \begin{itemize}
                \item 2개 이상의 더 많은 계를 한번에 표기할 때는, 복합계를 이루는 각각의 component system에 대한 tensor product를 사용한다.
                \item 각 component system의 state들의 tensor product로 나타낼 수 있는 상태를 product state라고 하며, product state로 표현할 수 없는 composite system만이 가지는 상태를 \alert{entangled state}라고 한다. 
                $$ \frac{1}{ \sqrt 2} (\ket{00} + \ket{11})$$
            \end{itemize}
        \end{frame}

        \begin{frame}{Dynamics in Composite system}
            \begin{itemize}
                \item $Q\otimes M$ 복합계에 작용하는 unitary $U$를 다음과 같이 가정하자.
                $$ \quad U\ket{\psi}\ket{0} = \sum_m M_m\ket{\psi} \ket{m} $$
                $\{M_h\}$, $\{M_m\}$은 각 component system의 measurement operator이다. 
                \item $U$는 다음과 같이 임의의 state $\ket \psi\ket 0, \ket \varphi \ket 0$에 대한 내적을 보존한다.
                \begin{align*} 
                    (\bra \varphi \bra 0 U^\dagger)(U \ket{\psi \ket 0}) &= \sum_{mm'} \bra{\varphi } \bra {m}  M_{m}^\dagger M_{m'}\ket{\psi} \ket {m'}  \\ &= \sum_{ mm'} \braket{\varphi|M^\dagger_{m}M_{m'}|\psi} \underbrace{ \braket{m|m'}}_{\delta_{mm'}} \\
                    &= \braket{\varphi|\Sigma_m M_m^\dagger M_m|\psi} \\ &= {\braket
                    {\varphi|\psi}} = \bra \varphi \underbrace{\braket{0|0}}_1 \ket \psi
                \end{align*}
            \end{itemize}
        \end{frame}

        \begin{frame}{Measurement in Composite system}
            \begin{itemize}
                \item 다음과 같은 projective measurement를 사용하면,
                $$P_{m} = I_a \otimes \ket m \bra m (= I_a \otimes P_{b,m})$$
                \item $U\ket \psi \ket 0$ 상태에 대해 outcome $m$의 확률은 다음과 같고
                \begin{align*} 
                    p(m) &= \bra{\psi}\braket{0|U^\dagger P_m U| \psi} \ket 0 \\ &= \sum_{m'm''} \braket{\psi|\bra{m'} M_{m'}^\dagger |(I_a \otimes \ket m \bra m) M_{m''} \ket{m''} \ket{\psi}} \\ &= \sum_{m'm''} \braket{\psi|I_a} \braket{m'|m} M_{m'}^\dagger M_{m''}\braket{m|m''} \braket{I_a|\psi} \\
                    &= \braket{\psi|M_m^\dagger M_m | \psi}.
                \end{align*}
                \item 측정 후 상태는 다음과 같이 변한다.
                $$ \frac{P_m U \ket{ \psi} \ket 0}{\sqrt{\bra \psi \braket{0|U^\dagger P_mU|\psi} \ket 0}} = \frac{M_m \ket{\psi} \ket m}{\sqrt{\braket{\psi|M_m^\dagger M_m |\psi}}}  = \frac{M_m \ket{\psi}}{ \sqrt{\braket{\psi|M^\dagger_m M_m | \psi}}}. $$
            \end{itemize}
            $\Rightarrow$ 즉, identity matrix $I$와 component system에만 작용하는 measurement operator $\ket m \bra m$의 tensor product로 정의되는 operator는 composite system에 작용하지만, component system에 작용하는 측정과 동일하다.  
        \end{frame}
    \end{section}

    %#################################### 
    \begin{section}{Density matrix}
        \begin{frame}{Density matrix}
        \end{frame}

        \begin{frame}{Rewrite postulates via density matrix}
        \end{frame}

        \begin{frame}{Reduced density matrix}
        \end{frame}
    \end{section}

    %#################################### 
    \begin{section}{Schmidt decomposition}
        \begin{frame}{Schmidt decomposition}
        \end{frame}

        \begin{frame}{Purification}
        \end{frame}
    \end{section}

    %#################################### 
    \begin{frame}{References}
        
        \begin{itemize}
            \item M. A. Nielson and I. L. Chuang, Quantum Computation and Quantum Information
            \item Lecture notes for QU511: Quantum Computing (Fall 2024)
        \end{itemize}
        \vspace{6cm}
        % \begin{alertblock}{test block}
        % \end{alertblock}
    \end{frame}
\end{document}

